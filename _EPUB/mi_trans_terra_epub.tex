\title{Os comedores de terra}

\title{Pitawarewë}

\textit{Ou o livro das transformações contadas pelos Yanomami do  grupo Parahiteri}

Anne Ballester (\textit{organização e tradução})

2ª edição

\chapter{}

\textbf{edição brasileira©} Hedra 2022\\
\textbf{organização e tradução©} Anne Ballester\\

\textbf{coordenação da coleção} Luísa Valentini\\
\textbf{edição} Luisa Valentini e Jorge Sallum\\
\textbf{coedição} Suzana Salama\\
\textbf{assistência editorial} Paulo Henrique Pompermaier\\
\textbf{revisão} Luisa Valentini, Vicente Sampaio e Renier Silva\\
\textbf{capa} Lucas Kroëff\\

\textbf{\textsc{isbn}} 978-65-89705-68-0

\textbf{conselho editorial} Adriano Scatolin, Antonio Valverde, Caio Gagliardi, Jorge Sallum, Ricardo Valle, Tales Ab'Saber, Tâmis Parron
 
\bigskip
\textit{Grafia atualizada segundo o Acordo Ortográfico da Língua
Portuguesa de 1990, em vigor no Brasil desde 2009.}\\

\vfill
\textit{Direitos reservados em língua\\ 
portuguesa somente para o Brasil}\\

\textsc{editora hedra ltda.}\\
R.~Fradique Coutinho, 1139 (subsolo)\\
05416--011 São Paulo \textsc{sp} Brasil\\
Telefone/Fax +55 11 3097 8304\\\smallskip
editora@hedra.com.br\\
www.hedra.com.br\\

Foi feito o depósito legal.

\chapter{}

\textbf{Os comedores de terra} apresenta histórias como o surgimento da maniva, a mandioca. E o dilúvio no qual Omawë --- irmão de Horonamɨ, grande pajé que surgiu dele mesmo e junto com as florestas ---, está envolvido e que deu origem aos brancos, os \textit{napë}. \textit{Os comedores de terra} faz parte do segmento Yanomami da coleção Mundo indígena --- com \textit{O surgimento da noite}, \textit{A árvore dos cantos} e \textit{Os comedores de terra} ---, que reúne quatro cadernos de histórias dos povos Yanomami, contadas pelo grupo Parahiteri. Trata-se da origem do mundo de acordo com os saberes deste povo, explicando como, aos poucos, ele veio a ser como é hoje.

\textbf{Anne Ballester} é coordenadora da \textsc{ong} Rios Profundos e tem experiência de vinte anos junto ao povo Yanomami do Rio Marauiá. Trabalhou como professora na área amazônica, e atuou como mediadora e intérprete em diversos \textit{xaponos} do rio Marauiá. Foi coordenadora do Programa de Educação quando ajudou a criar a \textsc{secoya} (1998), então comprometida com a garantia do sistema de saúde indígena. Dedicou-se à difusão da escola diferenciada nos \textit{xaponos} da região, como também à formação de professores Yanomami, em parceria com a \textsc{ccpy}/\,Roraima, incorporada atualmente ao Instituto Sócio Ambiental (\textsc{isa}). Ajudou a organizar cartilhas monolíngues e bilíngues para as escolas Yanomami, a fim de que os professores pudessem trabalhar em sua língua materna.

\textbf{Mundo Indígena} reúne materiais produzidos com pensadores de diferentes povos indígenas e pessoas que pesquisam, trabalham ou lutam pela garantia de seus direitos. Os livros foram feitos para serem utilizados pelas comunidades envolvidas na sua produção, e por isso uma parte significativa das obras é bilíngue. Esperamos divulgar a imensa diversidade linguística dos povos indígenas no Brasil, que compreende mais de 150 línguas pertencentes a mais de trinta famílias linguísticas.


\chapter{Apresentação}

Este livro reúne histórias contadas por pajés yanomami do rio Demini sobre os tempos antigos, quando seres que hoje são animais e espíritos eram gente como os Yanomami de hoje. Estas histórias contam como o mundo veio a ser como ele é agora.

Trata-se de um saber sobre a origem do mundo e dos conhecimentos dos Yanomami que as pessoas aprendem e amadurecem ao longo da vida, por isto este é um livro para adultos. As crianças yanomami também conhecem estas histórias, mas sugerimos que os pais das crianças de outros lugares as leiam antes de compartilhá-las com seus filhos.

\chapter{Como foi feito este livro}

\begin{flushright}
\textsc{anne ballester soares}
\end{flushright}

\noindent{}Os Yanomami habitam uma grande extensão da floresta amazônica, que cobre
parte dos estados de Roraima e do Amazonas, e também uma parte da
Venezuela. Sua população está estimada em 35 mil pessoas, que falam
quatro línguas diferentes, todas pertencentes a um pequeno tronco
linguístico isolado. Essas línguas são chamadas yanomae, ninam, sanuma e
xamatari.

As comunidades de onde veio este livro são falantes da língua xamatari
ocidental, e ficam no município de Barcelos, no estado do Amazonas, na
região conhecida como Médio Rio Negro, em torno do rio Demini. 

\section{Da transcrição à tradução}

Em 2008, as comunidades Ajuricaba, do rio Demini, Komixipɨwei, do rio
Jutaí, e Cachoeira Aracá, do rio Aracá --- todas situadas no município
de Barcelos, estado do Amazonas --- decidiram gravar e transcrever todas
as histórias contadas por seus pajés. Elas conseguiram fazer essas
gravações e transcrições com o apoio do Prêmio Culturas Indígenas de
2008, promovido pelo Ministério da Cultura e pela Associação Guarani
Tenonde Porã.

No mês de junho de 2009, o pajé Moraes, da comunidade de Komixipɨwei,
contou todas as histórias, auxiliado pelos pajés Mauricio, Romário e
Lauro. Os professores yanomami Tancredo e Maciel, da comunidade de
Ajuricaba, ajudaram nas viagens entre Ajuricaba e Barcelos durante a
realização do projeto. Depois, no mês de julho, Tancredo e outro
professor, Simão, me ajudaram a fazer a transcrição das gravações, e
Tancredo e Carlos, professores respectivamente de ajuricaba e
komixipɨwei, me ajudaram a fazer uma primeira tradução para a língua
portuguesa.  

Fomos melhorando essa tradução com a ajuda de muita gente: Otávio
Ironasiteri, que é professor yanomami na comunidade Bicho"-Açu, no rio
Marauiá, o linguista Henri Ramirez, e minha amiga Ieda Akselrude de
Seixas. Esse trabalho deu origem ao livro \textit{Nohi patama Parahiteri
pë rë kuonowei të ã} --- \textit{História mitológica do grupo Parahiteri},
editado em 2010 para circulação nas aldeias yanomami do Amazonas onde se
fala o xamatari, especialmente os rios Demini, Padauiri e Marauiá. Para quem quer conhecer melhor a língua xamatari, recomendamos os trabalhos de Henri Ramirez e o \textit{Diccionario enciclopedico de la lengua yãnomãmi}, de Jacques Lizot.  

\section{A publicação}
 
Em 2013, a editora Hedra propôs a essas mesmas comunidades e a mim que
fizéssemos uma reedição dos textos, retraduzindo, anotando e ordenando assim
narrativas para apresentar essas histórias para adultos e para crianças de todo
o Brasil. Assim, o livro original deu origem a diversos livros com as muitas
histórias contadas pelos pajés yanomami.  E com a ajuda do \textsc{proac},
programa de apoio da \textsc{secult--sp} e da antropóloga Luísa Valentini, que organiza a
série Mundo Indígena, publicamos agora uma versão bilíngue das principais
narrativas coletadas, com o digno propósito de fazer circular um livro que
seja, ao mesmo tempo, de uso dos yanomami e dos \textit{napë} --- como eles nos chamam. 

Este livro, assim como o volume do qual ele se origina, é dedicado com afeto à memória de nosso amigo, o indigenista e antropólogo Luis Fernando Pereira, que trabalhou muito com as comunidades yanomami do Demini.

\chapter{Para ler as palavras yanomami}


Foi adotada neste livro a ortografia elaborada pelo linguista Henri Ramirez, que é a mais utilizada no Brasil e, em particular, nos programas de alfabetização de comunidades yanomami. Para ter ideia dos sons, indicamos abaixo.

\bigskip

\begin{tabular}{rl}
/ɨ/ & vogal alta, emitida do céu da boca, próximo a \textit{i} e \textit{u}\\
/ë/ & vogal entre o \textit{e} e o \textit{o} do português\\
/w/ & \textit{u} curto, como em \textit{língua}\\
/y/ & \textit{i} curto, como em \textit{Mário}\\
/e/ & vogal \textit{e}, como em português\\
/o/ & \textit{o}, como em português\\
/u/ & \textit{u}, como em português\\
/i/ & \textit{i}, como em português\\
/a/ & \textit{a}, como em português\\
/p/ & como \textit{p} ou \textit{b} em português\\
/t/ & como \textit{t} ou \textit{d} em português\\
/k/ & como \textit{c} de \textit{casa}\\
/h/ & como o \textit{rr} em \textit{carro}, aspirado e suave\\
/x/ & como \textit{x} em \textit{xaxim}\\
/s/ & como \textit{s} em \textit{sapo}\\
/m/ & como \textit{m} em \textit{mamãe}\\
/n/ & como \textit{n} em \textit{nada}\\
/r/ & como \textit{r} em \textit{puro}\\
\end{tabular}


\chapter{Os comedores de terra}

Esta é a história dos nossos antepassados que aos poucos se
multiplicaram. Ela começa na época em que não havia Yanomami como os de
hoje. Os comedores de terra sofriam, porque eles comiam terra. Os
primeiros que surgiram sofreram. Nós também quase que teríamos sofrido,
como as minhocas, por cavar a terra e tomar vinho de barro, se não
fossem os acontecimentos que seguem. 

Só havia os comedores de terra. Eles não conheciam os alimentos que hoje
nos alimentam, apesar de serem muitos, como ingá, maparajuba,
conori. Havia cabari, bacaba, mas eles não sabiam tomar vinho de
bacaba; tomavam vinho de barro e de flores, depois de cortá-las lá em
cima, eles tomavam vinho delas. Devoravam as embaúbas novas, e as
chamavam de comida. Se nossos antepassados tivessem surgido nessa época,
nós estaríamos sofrendo hoje. 

Quem descobriu os alimentos comestíveis? Morava com eles Horonamɨ,
aquele cujo nome aparece no início, na origem. Ele mostrou a todos os
alimentos que até hoje nós comemos. Depois de perguntar, experimentar e
carregar os alimentos por todos os cantos, ele ensinou os Comedores de
Terra a comê-los. Foi ele, Horonamɨ, não outro. Assim foi.

Tudo isso não aconteceu embaixo deste céu, mas do céu que caiu\footnote{Queda do céu: Referência à história sobre a queda do céu anterior ao que existe hoje, contada no volume desta mesma série, \textit{O surgimento dos pássaros}.} e amassou
os primeiros habitantes. Abriram o céu e assim nossos antepassados
surgiram. O céu caiu, mas antes ele estava lá em cima, antes da
existência dos nossos antepassados, antes de algum \textit{napë},\footnote{O termo \textit{napë} designa os estrangeiros, em geral os brancos, ou quem adotou seus costumes.} entre
nós, perguntar assim:

--- Tudo bem? 

Eles morriam de fome, pois comiam terra, flores, frutas,
excrementos de minhoca, folhas novas de cabari. É essa a história dos
ancestrais. Os ancestrais no início não comiam os alimentos que comemos
hoje. Eles comiam a pasta que se forma nas árvores junto às casas de
cupim. Dizem que a comiam com voracidade. Apesar de só comerem isso, não
ficavam doentes, pois não existia malária, e não precisavam curar
ninguém, pois não havia doença, não havia dor, nem tosse, portanto não
havia necessidade de remédio --- não havia doença, pois não
havia \textit{napë}. Viviam bem, sem doenças, até terem muitos cabelos
brancos. As mulheres ficavam velhas até terem a cabeça branca, pois não
havia doença. 

Era assim, no início: não sofriam com conjuntivite, nem com feridas, nem
tinham marcas de furúnculo. Tinham a pele bonita e somente sofriam de
fome, por causa da terra que comiam. Apoiavam-se em paus para andar, por
causa da fome. Assim era. Nessa época, não sabiam comer carne, mas eles
estavam bem e, quando um velho morria, ninguém chorava. Não choravam por
causa de um velho morrendo de doença, pois ninguém morria de doença. Nem
havia cobra para picar, dar dor e matar. Eles viviam bem. Os espíritos
não pegavam a alma de ninguém para matar. Era assim. Eles não ficavam
fracos com diarreia, isso não acontecia, apesar de eles não tomarem
remédios. 

Era assim quando não existia \textit{napë}, antes de os \textit{napë} se
misturarem; nessa época, os \textit{napë} existiam? Sabemos que não! Não
existiam. 

Os rios, apesar de serem grandes, dizem que eram vazios. Dizem que não
se escutava o som de motor subindo o rio fazendo \textit{tu, tu, tu, tu, tu,
tu}!

\textit{Ũ, ũ, ũ, ũ, ũ}! Não se escutava o som do avião, por isso os velhos
não morriam de doença. Morriam de cegueira. Era assim que morriam, por
causa da cegueira. Tornavam-se cegos e a respiração parava, não por
causa de doença, mas de fome. Isso só aconteceria depois. Aconteceu
assim. 

Ninguém dizia:

--- Alguém lá pegou doença e morreu; eles estão chorando lá! 

Mesmo quando tinham cabelos brancos, eles andavam saudáveis. Morriam de
velhice. Ficavam cegos, os olhos secavam, o sangue acabava, por isso
morriam. Mandavam deixar os mortos fora do xapono,\footnote{Os xaponos são as 
casas coletivas circulares onde moram os Yanomami. Cada casa corresponde a 
uma comunidade; em geral não se fazem duas casas numa mesma localidade.} para que voltassem
como mortos-vivos. Retornavam sempre na forma de
mortos-vivos, quando não havia \textit{napë} entre eles. Os ancestrais
ficavam alegres por comer frutas, não era como agora. Quando comeram
carne, eles endoideceram e passaram mal. Não havia fogo e comiam cru.
Endoideceram por comer cru.

Depois de eles aprenderem a comer os verdadeiros alimentos, eles se
tornaram como nós. Tornaram-se assim, comendo carne cozida. Quando
aconteceu, as crianças se multiplicaram, saudáveis, em um e outro
xapono. Fizeram um grande xapono, outros se agruparam, e não
pararam de se multiplicar, todos saudáveis.

\chapter{Pitawarewë}

Yanomamɨ të yai përɨo mao tëhë, të pë rë përɨonowei të pënɨ kamiyë pëma
kɨ no patama rë wawërayonowei të ã xĩro. Pitawarewë pë kãi kua. Pita a
rë wanowehei, të pë rë no preaanowei, të kãi kua. Hapa, pata pë rë
kuonowei, ɨ̃hamɨ pë rë harayonowei, hapa pë no preaama. Hei pëma a ha
tɨɨanɨ, pëma u kɨ mori rë koanowei, horema kurenaha pëma kɨ no mori
preaaɨ kuoma. Pitawarewëteri, pë hiraoma kãi. Kama xĩro. Nii a taɨ maohe
tëhë, ɨ̃hɨ hei nii pruka a kua makure, kẽpo ũ kua makure, naɨ a kua
makure, momo a kua makure, të pë mohoti yaro. Wapu a kua makure, hoko ma
makui, mau kãi koaɨ taono rë mahei, hei a xĩro rë hurukuaɨnowehei. 

Horehore a ha të pë ha përahenɨ, horehore u hurukuamahe, kihi tokori kɨ
tuku rë xɨrɨkɨi, pë kãi wëhërɨmamahe, nii a wãha rë hiranowehei, ɨ̃hɨ
tëhë kamiyë iha pëma kɨ no patama pëtou ha kunoha, pëma kɨ no preaaɨ.
Ɨhɨ wetinɨ nii a yai waɨ rë tararenowei, kamanɨ pë kãi rë përɨonowei
hapa a wãha kua waikia. Ɨhɨnɨ nii a wahɨmaɨ piyëkoma. Hapa a wãha rë
kure, miha a wãha komosi rë prare. Ɨhɨnɨ nii a rë wamaɨwei pëma a wapë,
a ha wãriapotunɨ, kamanɨ a waɨ ha wapërënɨ, a rë yehinowei, kama
Pitawapëteri pë iha, a waɨ rë hirakenowei, ɨ̃hɨ a wãha kua. Hapa a wãha
kua kure, Horonamɨ. Ai tënɨ mai! Ɨnaha të kuoma. 

Hei a rë kui, hapa yai, të pë rii hirao tëhë, hei a ha kerɨnɨ, të pë rii
xëye hërɨma. Ɨhɨ ei a hamɨ, të pë rë pakakumanowei, të kuami. Hei a rë
kõhomope hamɨ, të kuami. Ei a rë kui, a rë tuyënowehei, hei a hamɨ
kamiyë pëma kɨ no patama pëtopë hei a kerayoma. A kerayoma, hei a rë
kui, kihamɨ a kuo parɨoma hapa. Kamiyë pëma kɨ mao tëhë, napë pë mao
tëhë, ai napë:

--- Wa totihiwë? --- të pë kuɨ nikereo mao tëhë, napë a kunomi.

Ohinɨ të pë rë nomanowei, a rë ixomanowehei, pita pë waɨhe yaro,
horehore pë waɨhe yaro, himohimo pë waɨhe yaro, horema xi u pë pata
koaɨhe yaro, wapu huhi hena pë pata tukunɨ, pë rë nomanowei, pë no rë
preaanowei, ɨ̃hɨ ei të ã hamɨ, pata të ã, hei të ã. Pata të pënɨ nii a
wãha waɨ haɨonomihe. 

Parɨwa të pë yainɨ nii a wãha wanomihe. Hei të wãha wamahe, ɨ̃hɨ të wãha
xĩro wamahe. Ɨnaha të pë wãha kuaama. Yupu uxi pë urihi hamɨ, uxi pë rë
yërëkëi, uxi pë ha hoyoahenɨ, uxi pë wãha wëhërɨmamahe. Hapa ɨ̃hɨ nii a
waimi makuhei, të pë kãi hurapɨonomi, hura a wayu kuonomi yaro, ai të ha
përɨnɨ, të pë nohi kãi rëayonomi, ai të ha ninirɨnɨ, të kãi niniri
kuonomi, tokomi a ha pëtarunɨ, õho! Të kãi kunomi, he horomamotima napë
a kuo mao tëhë, xawara a kuonomi yaro. Totihitawë hei kurenaha të pë ha
patanɨ, pëɨ mai të pë ha patanɨ, të pë he horoi kãi rë xaionowei, hapa
të pë kuoma. Të pë rë patayomai të pënɨ hawë horoi pë pata taapraramahe,
të pë pëimi yaro. 

Ai të ha mamoripɨrɨnɨ të mamo krihipɨ no preaanomi, të ha warapisirɨnɨ,
të kãi warapisi hunomi. Të pë ha yuupɨanɨ, të pë yuupɨ unosi kãi hunomi.
Totihitawë si kɨ kopehewë, ohi a wayu ha pita a ha, të pë no preaaɨ
tahiaoma. Ohi pënɨ të pë wãha payihoma. Ɨnaha të kuoma. Yaro a kãi waɨ
tao maohe tëhë. Ɨhɨ tëhë të pë totihitaa makure, hapa pata ai të ha
nomarɨnɨ, xomi xawaranɨ të pë kãi ɨ̃kɨɨ taonomi. Ɨkɨnomi. Ai të ha
ninirɨnɨ, kiha të ha tuyërarɨnɨ, të ha nomaruhurunɨ, të kãi kuaanomi
yaro. Totihitawë të pë hiraoma. Hekura pë kãi ixonomi, hekura pë makui,
të pë mɨ amo ha no uhutipɨ ha yuahenɨ, të pë kãi xëpranomihe, ɨ̃naha të
kuoma. Hapa krii, ai të pë ha kriiprarunɨ, të ëpëhëwë no kãi preaanomi,
të pë kuaaɨ taonomi, he horomamorewë kɨ koaimi makuhei. 

Napë a përɨo mao tëhë, napë pë nikereo mao tëhë, ɨ̃hɨ tëhë. Ɨhɨ tëhë napë
pë pruka hiraoma tao! Pë puhi kuɨ mai! Napë pë përɨonomi ha tarei, hapa
napë a kuonomi. 

Pata u kɨ makui, u pë hõra prokeoma. \textit{Tu, tu, tu, tu, tu, tu, tu, tu}! Ai
të mɨ yamou ha kuikunɨ, ai të hõra kuɨ hirionomi. 

\textit{Ũ, ũ, ũ, ũ, ũ}! Hei ai të ã kãi hirio mao tëhë, hõra kuonomi, kuwë
yaro, xawaranɨ, të pë rohote ha nomanɨ, të pë no preaanomi. Kama të pë
hupërɨnɨ, kama të pë hupërɨ ha rasio hërɨnɨ, ɨ̃naha xĩro. Kama të pë
hupërɨ, mixiã kɨ hawëoproma. Xawaranɨ mai! Ohinɨ. Kama të pë xomi wãha
nomama, hei të rë kure hamɨ, waiha të kuprarioma. Ɨnaha të kuprarioma.
Hei kurenaha: 

--- Ai të ha përɨnɨ, të nomarayou. Kiha ai të pë ɨ̃kɨtayou --- të pë
kunomi. 

Të pë henakɨ au makui, të pë huma temɨ. Ɨhɨ hapa të rë kuprarionowei, a
rohote nomarayoma, yërëariyoma. A hupërɨprarioma yaro, a krihipɨrayoma
yaro, ĩyë pë kãi maprarioma yaro, të rë kuprore tënɨ, të hesi kekei.
Kiha të tape kiriohe. Të ha tapa kɨrɨhenɨ, hei kurenaha të pë hõra
kõoma. No porepɨ të pë kõo parɨoma, napë a rë kui a nikereonomi. 

Të pë ã toprarotima, nii a waɨ ha tararɨhenɨ, të pë ã topraroma, kete a
wamahe, ɨ̃naha të pë kuaanomi makui, Yaroriwë pë wamahe, të pë xi wãrima,
wãrii no preoma. Kaɨ wakë kuonomi yaro. Riyë të pë wamahe. yëĩyë të pënɨ
të pë xi wãrima. 

Ɨhɨ kete a waɨ ha tararɨhenɨ, hei kurenaha, të pë kuprarioma. Yaro rɨpɨ
pë kãi waaremahe yaro. Rë kuprore tëhë, heinaxomi, të pë pararayoma,
temɨ, të pë parama, ai nahi ha të pë kãi parama, ai nahi ha të pë kãi
parama, ai të pë kãi parama, yahi pata a tamahe, pata ai të pë hiraoma,
mahu, hei mahu xapono hamɨ të pë pata otima, temɨ.

\chapter{A vingança de Horonamɨ}
 
Onde o sol se põe, naquela parte da floresta, foi naquela parte que se
transformaram. Os Cuatás viviam como Yanomami. Eles são gente.
Moravam como nós, na planície. Pica-Pau Vermelho morava junto com os
Cuatás, os Rapoahiteri e Lagartixa. Pica-Pau Vermelho e
Lagartixa, os salvadores de Horonamɨ, moravam com os Rapoahiteri.
Horonamɨ os encontrou, ele mesmo. Ele os viu comendo. Eles comiam abios.
Eles o chamaram, ardilosamente, para subir; provocaram o encontro
para fazê-lo gritar.

Horonamɨ subiu, eles o fizeram subir, subir… Eles o chamaram e
quando ele subiu e chegou bem no centro da árvore, em vez de comer, eles
puxaram outra árvore, como se estivesse amarrada, e enquanto desciam por
ela, disseram a Horonamɨ:

--- Fique aí comendo! Aí você ficará satisfeito! Coma essas frutas que
ninguém pegou! --- disseram eles.

Eles fizeram com que Horonamɨ ficasse ali. Cuatá, do abieiro onde
estava, puxou a árvore taxizeiro com o fio esticado e a ponta mal
encaixada, segurando-a somente pelas folhas. Todos os Cuatás saíram, e
aquele que eles tinham chamado ficou sozinho. Eles o deixaram preso no
abieiro. O taxizeiro deu um impulso. Ficou só o abieiro. 

Horonamɨ ficou agoniado, mesmo sendo Horonamɨ. Ficou gritando de cima,
ficou gritando, ficou preso lá. Quem iria buscá-lo? 

--- Quem virá me buscar? --- pensava ele, chorando. 

Agoniado, estava muito triste, gritava e pedia socorro;
os Rapoahiteri e os Cuatás o deixaram naquela
situação. Esse é o nome dos primeiros habitantes, os que viviam
na mesma época que os primeiros humanos, Rapoahiteri. Foram eles que o
deixaram ali agoniado. 

Bem depois, Lagartixa escutou os gritos de Horonamɨ. Lagartixa subiu,
para fazê-lo descer, queria buscá-lo, queria carregá-lo nas suas costas,
mas ele recusou, receando escorregar com ele. Horonamɨ estava com medo
de descer de cabeça para baixo com Lagartixa. 

--- Não, você não vai me fazer descer direito --- disse
Horonamɨ. --- Você vai me fazer cair! 

--- Vamos tentar! --- disse Lagartixa. --- Não tenha medo, eu não vou te
fazer cair! Eu te seguro bem forte! Coloque suas mãos, assim! 

Apesar de Lagartixa dizer isso, Horonamɨ tentou, mas os dois ficaram de
cabeça para baixo. Ele gritava, quase caiu de cima, Lagartixa quase o
fez cair e, como não dava certo, Horonamɨ desistiu. Quando ele desistiu,
porque infelizmente não dava certo, Pica-Pau Vermelho escutou a voz de
Horonamɨ: 

--- De quem é essa voz, de quem é essa voz? Parece a voz de alguém em
dificuldade --- disse. --- Alguém parece estar sofrendo mesmo, a voz,
qual é seu problema, \textit{ɨ̃ɨɨ}? --- disse. 

O Pica-Pau Vermelho chegou até lá e fez uma série de buracos, fez uma
espécie de escada no tronco da árvore. Ele fez a linha de buracos chegar
certinho à forquilha da árvore onde estava Horonamɨ. Ele mandou: 

--- Vai! Coloque suas mãos nos buracos e desça. Você não vai cair! Os
buracos estão prontos! --- disse. 

Ele fez muitos buracos, Pica-Pau Vermelho. Foi ele quem resolveu o
problema. Esses pica-paus são os que fazem buracos nas árvores. Foi ele
quem fez Horonamɨ descer, tirando-o daquela situação. É o nome dele
mesmo, Pica-Pau Vermelho, ele era gente. Graças à ação dele, os nossos 
antepassados se reproduziram e se multiplicaram. Foi assim. Pica-Pau Vermelho 
não era um animal, era um Yanomami. Ele existia como Yanomami e foi ele que fez Horonamɨ descer. 

--- Não responda mais, nem sempre você encontrará alguém para te
ajudar, tome outro rumo quando alguém te chamar! --- ele aconselhou a
Horonamɨ. 

--- Sim, você é meu amigo, eu gosto mesmo de você, vou te proteger, eu
não vou te fazer mal! --- agradeceu Horonamɨ. 

Depois disso, como vingança, Horonamɨ estragou nossos alimentos, ele nos
fez comer alimentos amargos, ele tornou os alimentos estranhos, nos
anestesiou a boca para os alimentos comestíveis, fez nosso paladar
estranhar outros alimentos. Ele enfiou uma flechinha envenenada nos
alimentos, enfiou em todos. 

O monstro Kuku devorou Horonamɨ porque ele agiu assim, ele
estragou todos os alimentos dessa forma. Ele os tornou amargos. 

No início, eles comiam cabaris crus, quando eram saborosos, pois não
eram amargos, antes de ele os envenenar.\footnote{Para comer os cabaris, é 
preciso deixá-los muitos dias na água e pisá-los para remover seu 
veneno.} Eles os comiam e eram gostosos; assim, comiam cabaris gostosos como beiju. Simplesmente os cozinhavam;
cozidos, os cabaris eram comidos no mesmo instante. Apesar de eles serem
assim, depois de Horonamɨ os picar com a flechinha --- ele picou todas
as sementes das frutas com o veneno --- ele os tornou amargos, todos.
Foi o que ele fez, e mostrou para eles. Ninguém mostrou a ele os frutos
amargos, foi ele que os tornou amargos, picando-os com veneno. Quando
ele terminou de picar todas as frutas, ele avisou: 

--- Isto que vocês comem são cabaris --- disse Horonamɨ. --- São
cabaris, e eram bons, mas vocês não os prepararão mais como preparavam.
Depois de algumas noites, vocês pisarão em cima deles e eles ficarão sem
gosto, aí vocês vão buscá-los, vocês os comerão. Agora eles são
amargos! --- disse ele. --- A preparação vai demorar muitos
dias! --- acrescentou. 

Assim fizeram. Aconteceu. Não foi qualquer pessoa que estragou as
frutas, foi Horonamɨ quem surgiu primeiro e as estragou, não foi um
descendente dele. Depois de fazer isso, estragar os alimentos com o
veneno, a seta com veneno estragou a cutia, grudou ao rabo da cutia, e
está lá ainda. O rabo das cutias se tornou a seta da zarabatana de
Horonamɨ, foi o que ele fez, e a cutia sofreu muito. Horonamɨ fez isso
com todos os alimentos, aqueles que eram gostosos, as frutas que eram
gostosas. 

--- Não sobrou nenhum!

De fato, nenhum sobrou mesmo, por isso ele disse assim. Ele os picou com
um veneno muito amargo. Quando ele terminou, ele foi se acabar também:
Kuku o comeu. Assim que foi. Como Horonamɨ não ficava quieto, ele acabou
numa situação difícil. 

\chapter{Horonamɨ a no yuo}

Hekura të pë wãha hayuamopë, kamiyë pëma kɨ nohi patama rë përɨonowei të
urihi hamɨ të pë rë kuaanowei, të pë rë kirimayonowei të kãi kua. Pei
motoka yai rë keiwei hamɨ, ɨ̃hɨ të urihi pata hamɨ të pë xi yai rë
wãrihonowei, të urihi kuwë. Paxori Yanomamɨ pë hiraoma. Yanomamɨ kë pë.
Hei kurenaha pë hiraoma, yarɨ hamɨ pë hiraoma. Toromɨriwë xo kɨ
përɨpɨoma, mɨ hetuoma. Ɨhɨ Toromɨriwë, Rehariwë kɨ rë prukamopɨre, kɨ
përɨpɨoma, Rapoahiteri pë hiraopë ha. Ɨhɨ pe he haa piyërema ɨ̃hɨnɨ rë. Pë
iaɨ tararema. Apia kɨ wamahe. A nomohori tuamapehe, a nakapehe, e pë ã
he hamorayoma, a miomiopraamapehe, a nakaremahe. A nakaɨ ha kuikutuhenɨ,
a nomohori tukema. 

A tumape herɨmahe. A ha tupo hërɨnɨ, a tuo tëhë, amoamo hi ha a kutou
tëhë, a iatou tëhë, kama pënɨ e te hi rë ututupouwehei, hawë hëyëhë të
õkaoma, pë itorayo hërɨma. 

--- Wa iaɨ hëo! --- e pë kuma. Ɨha, a ta pëtɨa hërɨ! --- e pë kuma. 

Hei xomixomi kɨ rë kui, kɨ ta waakɨ! --- pë kuma. 

A hëamapehe. Kihamɨ ɨ̃hɨ Paxori kama kɨ no mananaepɨnɨ e nini hi pata rë
ututaimaɨwei, ĩhete të wai hão pëoma, pei hi henakɨ ha. Kama pë ha piyëa
haikiikunɨ, a rë nakarehei, yami a hëtarioma. A xi wãrimakemahe. Ɨhɨ
tëhë nini hi pata karërayo hërɨma. Yami apia hi pata hëtario ha, a no
preaprarou hëoma. 

Ɨhɨ Horonamɨ rë. A rarɨprarou hëketayoma. A miomioprao hëketayoma, a xi
wãriketayoma. Wetinɨ a kõapë? 

--- Wetinɨ ware a kõapë? --- a puhi ha kunɨ, a ɨ̃kɨtayoma. 

A nohi hõrioma, a puhi õkitii yaro, a komɨpraroma, a nohi nakao ha, a
taamamahe. Rapoahiteri pënɨ. Taprano kãi mai! Hapa të pë rë kukenowei të
pë wãha. Yanomamɨ pë rë kuo xomao mɨ hetuonowei të pë wãha, Rapoahiteri.
Ɨhɨ pënɨ a xi wãrimakemahe. 

Yakumɨ Rehariwënɨ a wã he harema. Rehariwë e ha tupo hërɨnɨ, a kãi rĩya
ha itonɨ, a rĩya ha kõatunɨ, pei yaipë ha a hore yehiamaɨ puhii makui, a
kãi yõhoprariopɨ ha, e no xi ɨmaoma, Rehariwë iha. A mɨ kãi no
wërëprario hërɨpɨ ha, e kirima:

--- Ma, ware wa kãi no itou kateheopɨmi --- e kuma. --- Ware a kemarɨhë!
Ɨhɨ pë wapaɨ --- e kuma. 

--- Kirii mai! Pë kemaimi, pë ãkɨkɨpouwë --- a kuma --- Ɨnaha imɨkɨ ta
kuiku! --- e kuɨ makui, ɨ̃hɨ e wapëo makui, a mɨ kãi wërëama. 

A miomiomoma, e mori ketayoma, a mori kemama, ɨ̃naha e kuaaɨ ha a
tɨrakema. Kama pei he hamɨ e itorayo hërɨma. Ɨhɨ a ha tɨrakɨnɨ, ɨ̃naha a
kuaaɨ tikowë ha, Toromɨriwënɨ a wã hirirema. 

--- Weti kë a wã, weti kë a wã no prea yaia kupiyei? --- e kutarioma --- Të
ã no prea ayaa rë yai piyeië, ɨ̃ɨɨ, weti naha wa të hõra kuhawë, \textit{ɨ̃ɨɨ?} ---
e kuma. Ɨhɨ e waroa piyëkema. Hei kurenaha e të ka kɨ ta hërɨma,
tuomamotima te hi pë ka rë kurenaha e te hi ka kɨ pata tape herɨma.
Ɨhamɨ katitiwë e të hi ka kɨ waromaketayoma. Hi xerekeopë ha. A ximapë. 

--- Pei! Ɨhɨ ei të ka kɨ hamɨ wa imɨkɨ ha rukëkërukëkëmo hërɨnɨ, a ta ito
hërɨ! Wa kei mai kë të! --- e kuma --- Të ka kɨ kope waiki --- e kuma. 

Ɨhɨ a prukarema, Toromɨriwë a prukarema. Ɨhɨ a yainɨ a yokëmarema.
Toromɨ pënɨ hii hi pë ka rë taɨwehei, ɨ̃hɨnɨ a itomarema. A yokëmarema.
Ɨhɨ a wãha yai. Toromɨriwë, Yanomamɨ a kuoma. Ɨhɨ të pë uno hamɨ, kamiyë
pëma kɨ no patama, rarou hëaa hërɨma, pë paraɨ hëaa hërɨma. Ɨnaha të
kuoma. Toromɨriwë yaro a kuonomi, Yanomamɨ katiti a kuoma, kamiyë ya rë
kurenaha. Ɨnaha kuwë a përɨoma. Ɨhɨnɨ a itomarema.

--- Pei a wã huo kõo mai! Ɨnaha kuwë, ai wa të prukaɨ kõo mai kë të! Wa
yaiataro --- e ku hërɨma. 

--- Awei! Ɨnaha rë! Ipa nohi wa yaroi, pë yai nohimaɨ, kahë pë kãi
nowamaɨ, pë taimi! --- a kuma. 

Ɨnaha a ha taprarɨnɨ, kamiyë Yanomamɨ pëma kɨ ni pehi pëma kɨ rë iaɨwei,
të pë koamipramapë, nii pë kãi wamou puhi mohotipramapë, ai të pë nii
wamou aka kãi porepɨpramapë, nii pë kãi hĩma, ruhu makɨnɨ, husu të kɨnɨ
të pë kãi hĩma, të pë hĩa xoarare herɨma. Kama ɨ̃naha a tapraɨ puhio
yaro, Kukunanɨ a wapë, nii pë wãrihĩto tapraɨ haikirayo hërɨma. Të pë
kõamiamapë. 

Hapa wapu pë rë kui, riyëriyë pë wamahe, pë mɨhɨtatio tëhë, kõami
kuonomi, hapa a wãrihĩto tapraɨ mao tëhë, mɨhɨtatiwë naxi hi wamou,
mɨhɨtatiwë kurenaha pë wamahe. Harii pëo, rɨpɨpraɨ, waɨ katitoma. Kuoma
makui, kuaama makui, ruhu ma pë husunɨ të pë hĩma, kete moroxi pë rë
kutarenaha moroxi pë hĩi kuaama. Hɨtɨtɨwë të pë kõamiprarioma. Ɨnaha a
kuprarioma. Të pë wãhɨmama. Ai tënɨ e të wãhɨmanomi, kamanɨ të pë rë
kõamipraɨwei, të pë hĩma yaro. Ɨhɨ he usukuwë të pë hĩi hëwëkema yaro.
Të pë yɨmɨkamaɨ xoaoma, të pë yɨmɨkamarema. 

--- Pei! Hei të kɨ rë kui, wama të kɨ waɨ ha, hei wapu pë wãha! --- a kuma
--- Hei wapu kë pë, a totihitaoma makui, ɨ̃hɨ wama pë rë tahe naha, wama
të taɨ kõo maopë! Ɨnaha wa të kɨ titi ha taprarɨnɨ, hei tëhë wama kɨ
karukaɨ, kɨ okepropë, hei të ha wama pë kõapraɨ, wama pë wapë, pë
kõamipraruhe! --- e kumahe --- Të kɨ titi pata wai tetehepramaɨ
xoarayou? --- e pë kuma. 

Ɨnaha të tamahe. Të kuprarioma. Hei ai të ha përɨtarunɨ tënɨ nii pë
wãrihĩto tanomi, ɨ̃hɨ hapa a rë pëtarionoweinɨ të pë wãrihĩto tarema, ɨ̃hɨ
ai notiwa tënɨ të pë wãrihĩto tanomi. Ɨhɨ të ha taprarɨnɨ, pë wãrihĩto
tarema yaro, tomɨ iha, ɨ̃hɨ rë ma pënɨ wãrihĩto rë tanowei, tomɨ texina
hamɨ e ma xatia xoaa kure, ɨ̃hɨ rë pë texina, ɨ̃hɨ rë e ma ruhu kë ma.
Ɨnaha a kãi të taprarema, a hamiri no preaamama, ɨnaha a ha taprarɨnɨ:

Pei nii a rë totihitaohe, kete a kãi totihitaoma makui, ɨ̃naha a kuwë
haikiprarioma. 

--- Ai të hëami! 

Ai të hëpranomi yaro a kuma. A kuɨ xoaoma. Kõami të kɨnɨ të pë husunɨ të
pë hĩma yaro. Ɨhɨ të pë he ha wëprarɨnɨ, wëpraɨ katitio tëhë, kama a waa
tikorema. A warema. Ɨnaha a kuprarioma. Kama a yanɨkɨonomi tikoo yaro, a
no preaama.

\chapter{Como morreu o monstro Kuku}

Depois de tudo que nos ensinou, Horonamɨ acabou morto pelo monstro Kuku.
Sua mulher estava no final da gravidez e, quando ela sentiu as primeiras
dores do parto, o monstro matou o pai. Quem esfregou a barriga para a
criança nascer rapidamente foi Yoahiwë ou Yoawë, o irmão mais velho de
Horonamɨ. 

A partir do momento que o tuxaua Horonamɨ sumiu, Yoahiwë soprou a
montanha, que era a casa dos espíritos, pois queria matar o monstro Kuku
e vingar a morte de seu irmão. 

Ele fez um tipo de arma. Por ser de pedra, a montanha era
indestrutível. 

O monstro Kuku guardava os ossos de Horonamɨ dentro da montanha e os
devorava quando a criança nasceu. Nascida a criança, Yoahiwë a pegou com
a placenta e a lavou em água limpa. O tio pegou logo a criança
recém-nascida e a soprou para secá-la e acalmá-la. Enquanto isso, ele
preparava a zarabatana e escolhia as pedras. Fazendo isso ele nos
ensinou a matar. Ele conseguiu vingar Horonamɨ. 

A criança o fazia se lembrar do irmão bonito que lhe fora arrancado
enquanto ele a mantinha deitada sobre seu peito:

--- \textit{Ũa, ũa, ũa} --- fazia a criança. 

A criança não havia deitado com a mãe e nem mamado ainda quando Yoahiwë
soprou fortemente sua boca. Perto, havia um cipó pendurado, um cipó bem
duro, que ele torou; amontoou e amarrou muitas pedras, ele escolheu uma
pedra bem grande e volumosa, colocou-a na zarabatana e fez a criança
soprar. Apesar de a criança ser pequena, saiu um sopro forte. 

--- Meu filho, teu sopro já é forte! --- falou. --- \textit{Kuxu}, \textit{kuxu},
\textit{kuxu}! --- fez para a criança, que estava sentada nas coxas do
tio. --- Vamos, meu filho, já fortaleceu teu sopro? --- perguntou.

--- \textit{Hɨhɨ}! --- assentiu a criança. 

--- Experimente! Experimente com isso! 

--- \textit{Hɨhɨ}! 

Ela ficava de pé vacilante, como os filhotes de jacamim. 

--- \textit{Kuxu}, \textit{kuxu}, \textit{kuxu}! Fique firme, fique firme! --- o tio apoiava a
criança contra seu peito. 

--- Tente! Tente! 

Apesar de ser recém-nascida, \textit{paha}! Ela não soprava devagar. 
   
Ele segurava a criança na cintura, apoiando-a contra seu peito. Ela 
fazia as pedras se soltarem com um som forte, parecido com o som dos
conoris quando abrem. O cipó-de-apuí representava a imagem do monstro
Kuku que ele iria mesmo matar. 

\textit{Tëɨ̃ɨɨɨɨ}! O cipó se destruiu em pedaços. 

--- Vamos! Outro, outro, outro, só mais um!

Ela soprou novamente. \textit{Paha}! Ouviu-se o som. O meio do cipó explodiu em pedaços. 

--- Bem, o teu sopro já é forte. --- Ele fez explodir o pedaço de cipó
que sobrava. Ele riu. Ela foi perseguir o monstro, essa mesma criança
que havia nascido naquele dia, de manhã cedo. 

--- \textit{Hoaa}, é mesmo o meu filhinho! 

Pegou a criança nos seus braços para vingar o pai dela. Apesar de ser
pequena, a criança vingou seu pai. 

Apesar de a montanha ser dura, ela resistiu? Não! A criança fez explodir um
pedaço da serra, pois as pedras eram duras. 

\textit{Paha}! A criança fez cair a serra no chão em um monte de pedaços. Os
pedaços de pedra zoavam.

\textit{Tuuuuu}! Ela fez zoar os pedaços de pedra. Enquanto isso, o monstro Kuku
chorava de medo. Ele se lamentava, enquanto eles se aproximavam. Ele
chorava muito:

--- \textit{Ɨ̃ɨɨɨ}! O que vai ser de mim? --- ele gemia, assustado. 

Os últimos pedaços da montanha ficaram pendurados lá. O último pedaço
caiu com o monstro e o destruiu: 

--- \textit{Ku}! \textit{Ku}! \textit{Ku}! \textit{Ku}! --- fez o monstro. 

Embora o monstro estivesse morrendo, ele conseguiu matar o bebê. Com
muita dor pelo irmão bonito, que o monstro conseguiu extinguir quando a
criança nasceu, o tio a deitou sobre seu peito para fazer dela o
instrumento da sua vingança. Yoahiwë ficou com muita raiva da morte do
seu sobrinho, a quem tinha se apegado como se fosse seu próprio filho,
iludindo-se com a ideia de criá-lo. Como perdeu seu sobrinho, o irmão
mais velho de Horonamɨ fugiu rapidamente num tipo de jangada e, enquanto
fugia, transformou-se em espírito. Seus dois irmãos Yoahiwë e Omawë correram e seguiram pelo rio, zangados. 

--- \textit{Aë, aë, aë, aë, aë, aë}! --- dizia ele sem parar, de raiva pela morte
do seu sobrinho. 

Sim. Assim fizeram.

\chapter{Kuku na}
 
Hekura a rë pëtarionowei, ya të koro rë prakɨhe, ɨ̃hɨ katehe Horonamɨ a
wãha kuoma. Ai të përɨo mao tëhë, a përɨo rë parɨonowei, hapa ɨ̃naha të
kuprarioma. Ruwëri a xëprarema, a ha xëprarɨnɨ, të pë mioma, kamanɨ të
pë kãi përɨawei, ohi a wayunɨ, titi a mɨ haruu ha maikunɨ, titi rape të
kua makure, hei pëma kɨ he ou. Ɨhɨ titi tute a kuo tëhë, pë ka
hẽhaprarema. Ɨhɨ a rë mataruhe tëhë, pata a noã ha, kɨ rë horanowei, pei
makɨ hekura pë yahipɨ horama, Kuku kë na xëpraɨ puhiopë yaro, a no
mɨhɨapë, e makɨ horama. 

Mokawa pë rë kurenaha të taprarema. 

Maa ma kɨ makui, e makɨ kãi yãxikonomi. Ɨhɨ patanɨ, ihirupɨ e makasi
poyakawë kuoma. A mori payeriama. Ɨhɨ rë ihirupɨ makasi kuprarioma,
Horonamɨ ihirupɨ. Pë ihirupɨ e makasi wayu mori kuprou ahetou tëhë, e
warema. Ɨhɨ e kepraɨ tëhë, a no yuapë, ihirupɨ e makasi hokokama, makasi
paɨsama. Pata e yai wãha rë kuonowei, ɨ̃hɨ rë e wãha Yoahiwë kuoma. Yoawë
e wãha kuoma, Horonamɨ patapɨ yai. Pë pata a kãi yai rë përɨonowei, ɨ̃hɨ
rë a wãha Yoawë kuoma. 

Ɨhɨ ihirupɨ e keprarema, kiha, pë hɨɨ e ũ pë tapohorayoma, pei makɨ ka
ha, Horonamɨ ũ pë titioma, ɨ̃ha a waharayoma; a ha warɨnɨ, ihirupɨ e wai
keprarioma. yëĩyë pë hɨɨ patanɨ e maoa nokarema, a wai horaximama. A kuaaɨ tëhë, a rë horaɨwei e të si ka kopemama, masiri kɨ yaima. Maa ma
komore kurenaha. Kutaenɨ, napë të pënɨ exi të tapraremahe, të pë rë
niayore? Ɨhɨnɨ të pë rë hiranowei, të taprarema. Waiha a no yuaɨ he
yatiopë, Horonamɨ katehe a kuoma yaro. Pë oxe pë no xi hiraa he
yatimarei yaro, të wai kepraɨ tëhë, pë hɨɨ pata parɨkɨ ha e wai
makepoma:

--- \textit{Ũa, ũa, ũa} --- e të wai kuma. 

Të wai nokarema pë hɨɨ patanɨ. E të wai yaruprarema. E amixi kãi kõo
parɨonomi. Hei a rë nokaare, kama e u ha wai a yarurema. A wai
tikëmakema, a haximarema. Yonoupë pë kãi nokarema. Pë nɨɨ iha a
yakaamanomi. Mixiã kɨ hiakawë hipëoma, kama ihirupɨ iha. E mixiã kɨ
titihoma. Ɨhɨ ainɨ masiri komore, maa ma kurenaha, wãtarakawë raperape e
kɨ yaima. Kiha hiakawë totihiwë hawë ãrokoto pë pata rë kure, ɨ̃hɨ e të
pë pata përema, mixiã kɨ wapamapë, hei masiri kɨ pata xɨrɨkamaɨ
piyëkoma, masiri kɨ pata kõkaprarema. Ɨhɨ të kɨ pata yutuhamaɨ piyëkou
puhiopë yaro. A wai oxe makui, e të mixiã kɨ hiakawë hatariyoma. 

-Ɨhɨ rë kë, xei, wa mixiã kɨ wai hiakaprario kuhe? Kuxu, kuxu, kuxu,
hiakaprou ta haɨro! --- pë hɨɨnɨ ɨ̃hɨ rë të wai tikëmaporanɨ, e kuma. ---
Pei xei, wa mixiã kɨ wai hiakaprario kuhe?

--- \textit{Hɨ̃ɨ}! --- e wai kuma. 

--- Wapëpraa! Të ta wapëpraa! 

--- \textit{Hɨ̃ɨ}! --- e të wai kuma. 

E të wai uprao yatitiwë makui, yãpi pë ihirupɨ wai rë kurenaha, e
kuaama. 

--- \textit{Kuxu, kuxu, kuxu}! Hiakataru, hiakataru! --- Pë hɨɨnɨ pata parɨkɨ ha a
hamapoma. --- Pei, wapëpraa, wapëpraa! 

E keano tute makui, \textit{paha!} E të opi horaaɨ ma rë mare! Ihirupɨ pëɨxokɨ
wai huwëporanɨ, a wai hamapoma. Hawë momo e kosi homopramaɨ xoarayoma.
Momo kosipë homoprou hiakawë rë kurenaha, e të mixiã kɨ kuma. Hei
ãrokoto pata rë kui no owa rë kui, Kuku na no uhutipɨ pata rë tapo
piyei. 

--- \textit{Tëɨ̃ɨɨɨɨ}! --- hemata pata yutumarema. 

--- Pei, ai a, ai a, hɨtɨtɨ a --- të mixiã kɨ wai kutou kõrayoma 

\textit{Paha!} Pëɨxokɨ totosinɨ të pata yai yutumarema 

--- Pei, wa mixiã kɨ wai hiakaprou waikirohe! --- e kuma. E toto hemata
pata yutumarema. E ka ĩkapraroma. Ɨhɨ rë a kãi ukuo xoaa kure, hei a wai
rë keprou henare tëhë. 

--- \textit{Hoaaa!} Xëtëwë rë kë ë! 

Pë hɨɨ a no yuamapë, e wai maore hërɨma. Oxe makui a no yurema,
ihirupɨnɨ. 

Pei makɨ hiakawë makui, e makɨ hiakao ma rë kë? Të kɨ pata ãtahu
yutuhamama, masiri kohipɨ pë yai yaro. Napë pënɨ të pë rë taɨwehei naha
kutaenɨ, ɨ̃hɨ të pënɨ të pë niayou, të hiraɨ ha të tama. Të pë rasisiwë
ukëo ma rë mai kë! A kãi rë ukuore, hei naxomi naha a wapaɨ waikia
kurahari, hei a wapaɨ waikia. 

\textit{Paha!} Ɨhɨ pë hɨɨnɨ a tikëmaporanɨ, pë hɨɨnɨ hapa e kɨ pehi horakema, a
hiraɨ ha. Ãtahuãtahu komi të kɨ mɨ pata tikukoma. Pei të makɨ ãtahu pata
tiririmoma. 

--- \textit{Tuuuuu!} --- të kɨ ãtahu pata tatamama. Ɨhɨ kë kɨ taamaɨ ha, Kuku nanɨ
pë hɨɨ a rë warenowei, Kuku na pata ɨ̃kɨma. Kuku na pata kiriri ɨ̃kɨma. A nohi hõrio ha, të aheteprou waikirayou ha. E të pata ɨ̃kɨɨ:

--- \textit{Ɨɨɨɨ}! Kamiyë kë! --- e pata kuma, kiriri. 

Ɨhɨ kihi heaka hamɨ kɨ pata rë reiamarati, hɨtɨtɨ ɨ̃hɨ kihi rë të kɨ pata
rë hëtaruhe, të pata patikama kiriopë. Kuku na pata patikimaye kirioma:

--- \textit{Ku}! \textit{Ku}! \textit{Ku}! \textit{Ku}! --- të pata tamama. 

Ɨhɨ no pëxɨrɨ rë taare ha, hei ihirupɨ patanɨ a rë yurehe, a puhi no
ãxokei yaro, ɨ̃hɨ rë të mɨ haru ha, e rë hi hatoprare, imi huherayou
yaro, pë hɨɨ pata e xi rë iha rë wãrihiprouwei, ɨ̃ha rë e rii tokua
xoarayoma, e pehi yëa xoarayo hërɨma, hõrohõro kɨ hamɨ e pehi yëa
xoarayo hërɨma, hõrohõro kë kɨ tiyëwa ha e pehi kãi yëa xoarayoma. 

--- \textit{Aë, aë, aë, aë, aë, aë}! --- e kurayo hërɨma, ihirupɨ a no pëxɨrɨ ha. 

Awei, ɨ̃naha të pë kuaama. 

\chapter{Omawë}

Esta história começa com o nome dos Hoaxiwëteri. Omawë e Yoasiwë moravam
com os Hoaxiwëteri. O tuxaua Hoaxi, Caiarara, convivia com eles, por isso se
chamavam Hoaxiwëteri. Nesse mesmo lugar, junto com os Hoaxiwëteri,
moravam Omawë, que era o irmão mais novo, e Yoasiwë, o mais velho.
Omawë, mais novo, nasceu depois de Yoasiwë. Então, Yoasiwë e seu irmão
mais novo, Omawë, moravam com os Hoaxiwëteri.

Eles pegaram a filha do monstro Raharariwë. Os dois viram a filha de
Raharariwë. Totewë, outro nome de Yoasiwë, viu a filha de Raharariwë
sentada, pegando piabas --- ensinando assim a pegar piabas. Yoasiwë
desceu ao rio e, apesar de ele não ter anzóis, onde estava pegando
piabas, ele fez aparecer um tipo de anzol. Não existia anzol, mas com
seu pensamento, ele o fez surgir e o amarrou em uma espécie de gancho de pau.

Raharariwë e suas filhas moravam no rio Tanape. Era lá que ficava o
xapono de Raharariwë. Omawë as encontrou no rio Tanape. 

Era a própria filha de Raharariwë que se chamava Tepahariyoma, Mulher Matrinxã, 
aquele peixe branco, de rosto bonito. A irmã mais nova se chamava Peixe. 

No início, quando não havia mulher nos outros xapono, quando não existia
mulher entre os homens, os dois pegaram e levaram a filha mais velha de
Raharariwë. O irmão mais novo, que era lindo, conseguiu pegá-la, embora
os dois fossem anambés-azuis muito bonitos. 

--- Que passarinho bonito! --- Vocês dizem assim, pois Omawë era bonito.
Foi para ele que a mulher se entregou. 

O irmão mais velho de Omawë se chamava Hëɨmɨriwë, Anambé-Azul, queria se transformar em
anambé-azul;\footnote{O anambé-azul fornece penas azuis usadas para fazer os brincos dos
pajés.} o nome do irmão mais novo era Omawë,
saíra-paraíso, bonito, para conseguir pegar as duas mulheres, foi ele
quem fez as duas mulheres se levantarem. Depois de os dois pegarem essas
duas mulheres, não perguntem o que aconteceu! 

Omawë e seu irmão mais velho, Yoasiwë, o menos bonito, não foram ao rio,
pois estavam em um lugar diferente. Não moraram logo no lugar onde
pegaram as filhas do monstro aquático Raharariwë: foi a imagem deles que
se deslocou, na forma de passarinho.

Omawë não andava na terra, a história de Omawë é essa, ele não
andava na terra para pegar mulher, pois queria se tornar espírito. Ele
não criou os Yanomami. Ele era sozinho, independente, pois queria se
tornar eterno. A imagem dele ainda chega aos Yanomami, é ele que se
chama Omawë.

\chapter{Omawë}

Hei Hoaxiwëteri pë wãha kua, ɨ̃hɨ kɨpɨnɨ, Omawë, Yoasiwë pata xo ɨ̃hɨ
kɨpɨnɨ pë kãi përɨoma, ɨ̃hɨ Hoaxiwëteri. Kama Hoaxi a nikereoma yaro, ɨ̃hɨ
kama Hoaxiwëteri pë wãha kuoma. Ɨha ɨ̃hɨnɨ pë kãi rë përɨonowei Omawë
oxe, Yoasiwë pata, pata e yai kuoma, pata a kãi yai përɨoma. Omawë oxenɨ
Yoasiwë a nosi poma. Kutaenɨ Hoaxiwëteri, Yoasiwë, Omawë oxe xo pë kãi
përɨpɨoma. 

Raharariwë tëëpɨ rë yupɨrenowei. Rahariwë tëëpɨ ha tapɨrarënɨ, Totewë
patanɨ Raharariwë tëëpɨ kãi roa tararema. Raharariwë. Yaraka kɨ rëkaɨ
hiraɨ ha, ɨ̃hɨ Yoawënɨ e kɨ rëkaɨ mɨ wërëkema. Ihiya pë kãi kuami makui,
ɨ̃hɨ kɨ rëkapë ha, ihiya e kɨ pëtarioma. Ihiya e kuonomi. Kama puhinɨ
ihiya a pëtamarema, hii hi yõpënama ha a wai rë hĩikenowei. 

Raharariwë tëëpɨ kãi rë përɨonowei kama pë hɨɨ a rë përɨonowei, Tanape u
ha tëëpɨ kãi përɨoma, Raharariwë Tanape u pata ha. Kama xaponopɨ praoma.
A he harema, e kɨ he ha hapɨrënɨ, Tanape u pata ha. 

Raharariwë tëëpɨ yai, tëëpɨ yai, ɨ̃hɨ tëë e yai Tepahariyoma e kuoma, ɨ̃hɨ
e wãha. Ɨhɨ tepaharimɨ rë wama xixi pë waɨ, pë ma rë auwei, të pë mohekɨ
ma rë totihitai. Oxe e rë kui, Marohayoma e wãha kuoma. 

Suwë pë mao tëhë, ai a yahi hamɨ suwë ai a nikereo mao tëhë, hapa
Raharariwë tëëpɨ yupɨrema, e yëpɨpɨrema. Oxenɨ Omawë katehe a yainɨ, a
yëpɨrema Omawë pata Hëɨmɨriwë katehe kɨpɨ yaro. 

Kihi kiritamɨ a totihitawë! Wama kɨ rë kuɨwei, Omawë katehe e kuoma. Ɨhɨ
a yainɨ, suwë a no xi ɨhɨtarema. 

Hëɨmɨriwë pata a wãha kuoma, Omawë oxe katehe e wãha kuoma. Kɨ wã kãi rë
hokëpɨpramarenowei ɨ̃hɨ e yai. Ɨhɨ weti naha suwë kɨ ha yupɨrënɨ kɨ
kupɨprono mai! 

Mau u hamɨ, Omawë, Yoasiwë totexi pata xo kɨ hupɨnomi, kɨ rii yaipɨa.
Ɨhɨ kɨ rë yupɨrenowei, Raharariwë tëëpɨ rë yëpɨpɨrenowei ɨ̃ha kɨ përɨopɨo
xoaonomi. Ɨhɨ Marohayoma iha, Hoaxiriwë mo ha wëtɨpramarɨnɨ, të mɨ haru
ha, kihamɨ, kɨpɨ heakaprou xoarayoma, heaka hamɨ rii. Ɨhamɨ pë tëë e kãi
yure hërɨma. 

Pepi hamɨ, Omawë a huɨ kõo taonomi. Omawë të ã rë kui, pita hamɨ Omawënɨ
suwë katehe a rë yurenowei a hunomi, a huɨ taonomi. Omawë a rë kui kama
a hekura hupë, Yanomamɨ të pë tapra hërɨnomi. Kama yami a yaia, parimi a
kuprou puhiopë yaro. A no uhutipɨ warou maprou rë mai, pei a no
uhutipɨnɨ Yanomamɨ të pë rë haika rë xoare, ɨ̃hɨ a yai wãha Omawë. 

\chapter{O surgimento da maniva}
 
Quem pegou e espalhou a maniva para nós comermos? Não havia maniva.
Nossos antepassados não possuíam maniva como a que nós plantamos. Não
tinham maniva, ela não existia. 

No início, quando o monstro Raharariwë morava com suas filhas, sem outros
parentes, seu xapono ficava no fundo da água. Raharariwë não morava em
terra seca. No início, ele tinha sua casa dentro da água. 

As manivas estavam em uma casa que tinha a estrutura fincada, como a
nossa. Ela tinha esteios de maniva fincada. Ele possuía maniva. Morava
com elas, Raharariwë, ele mesmo. No início, era ele que a possuía, a
maniva que foi multiplicada e dividida. Quando ninguém tinha plantação
de maniva, ele amarrava sua rede lá onde ela estava. Ele guardava essa casa
de maniva fincada. Se não fosse ele, nós sofreríamos, se Raharariwë não
possuísse a maniva. 

--- Vou plantar maniva. --- A gente nem diria isso. A partir de
Raharariwë, a maniva se espalhou e nós comemos farinha. A maniva se
espalhou como alimento. Ele possuía a maniva, aquele cujas filhas foram
pegas. Omawë a quebrou e a pegou.

Quem foi traiçoeiramente chamado para dentro dessa casa? Quem chegou? A
filha entrou com ele dentro da casa de Raharariwë. Ela
entrou maliciosamente com ele dentro da água. A filha estava lá de pé.
Olha só a água! Olhe a superfície da água! A filha estava de pé dentro
da água escura. Ela o levou maliciosamente, para que o pai o assustasse.
Ele assim conseguiu chegar até Raharariwë, mas só para passar medo. 

Chamava-se Omawë. Ele tinha o nome daquela saíra verde tão bonita, não
de outro passarinho! Foi ele mesmo, Omawë, que levou a filha de
Raharariwë. Ele conseguiu entrar na água depois de fechar as narinas,
foi sua mulher que as fechou:

--- \textit{Kopou}! \textit{Ũi, ũi, ũi, ũi, ũi, ũi}! --- ela fez assim. --- Você não se
afogará! --- disse ela. 

--- \textit{Hɨ̃ɨɨ}! \textit{Tëɨ}! 

O xapono logo ficou sem água. A casa estava pintada de vermelho escuro como
as frutas \textit{hũria} bem maduras, como o vermelho cor de sangue das
asas de papagaio em movimento. Raharariwë morava em uma casa bonita e
Omawë chegou até lá. A filha lhe falou:

--- Pai! É teu genro! Eu trouxe teu genro, eu casei! 

Raharariwë olhou para Omawë. 

--- Aquele é teu? \textit{Hɨ̃ɨɨ}! É teu?

--- Pai! É meu marido! Eu casei! Ele é bonito, tu não achas? 

Ele sorriu. Lá, aquelas raízes saídas da terra pareciam enormes.

Onde o sogro estava, havia um pau grande, enfeitado e bonito, no chão,
parecido com a pele das costas de um poraquê. A filha sentou com Omawë.

--- Vai te sentar naquele pau com teu marido, filha
querida! 

Ele o fez sentar em cima do pau \textit{iwaiwakana} para assustá-lo. Esse
pau se mexeu como se mexem os jacarés. Ele o fez sentar em cima de um
pau que se mexia por si só. Omawë se sentou e perguntou a ela: 

--- Isto é madeira?

--- Não é madeira! É maniva! São manivas! --- disse a mulher. 

Nós chamaremos de maniva, foi ela que ensinou o nome maniva. 

--- Como se faz aquilo? 

--- Ele quebra para fazer beiju, ele faz beiju e come --- respondeu
ela. 

--- \textit{Hoaa}! --- exclamou. 

Ele tirou algumas mudas de maniva que estavam juntas:

--- \textit{Kero}! Eu plantarei minhas manivas! --- Omawë disse reservadamente à esposa. 

Raharariwë quase comia o genro, se a filha não o protegesse,
guardando-o sentado no seu colo. 

--- Tu me pegas e me fazes sentar no teu colo. 

E assim outras farão, fazer sentar o marido no colo. Assim ela fez e
ensinou a fazer. 

Omawë ficou com medo por causa do pau que se mexia, ficou assustado,
gritou e chorou como uma criança. Chorou de medo. 

Como o pau estava se mexendo, Omawë se tornou caba e grudou ao pau, se
transformou em caba \textit{tutu si}, daquelas branquinhas. Ele ficou
abalado. 

Ela conteve o pai: 

--- Pai! Não faça isso! É meu, eu casei! Não faça isso, por favor! Se
você fizer isso, eu não ficarei aqui! Vamos voltar, nós dois!

Ela mostrou a frente da casa. Raharariwë quase comeu Omawë, por isso ela
fugiu com ele. Ela também fugiu, pois Raharariwë quase comeu o marido. 

--- \textit{Tu, tu, tu, tu, tu, tu}! --- ele já estava dizendo.\footnote{Aqui há um jogo com o verbo \textit{tu}, cozinhar em água.}
%Jorge, por favor verifique se funcionou a nota que inseri aqui 

Aquele que levou as manivas de Raharariwë boiou com elas e as deu aos
seus parentes. 

Nós comemos o que ele deixou para acompanhar nossa comida, para não
sentir mais fome, ele pegou para nós comermos. 

Ele levou as mudas de maniva, cujos brotos ficaram chorando. Já havia
realmente fugido. 

Levaram manivas cujos irmãos ficaram chorando. 

--- Pai! Pai! --- choravam assim. 

Assim fizeram.

\chapter{Naxi si të ã}

Ɨhɨ weti iha naxi si piyëamopë, si ha kuonɨ weti iha si piyëremahe? Ɨhɨ
kamiyë pëma kɨ iapë, naxi si kuonomi. Hei kurenaha pëma të pë ha
kearɨnɨ, pëma të pë tapë, pata të pë rë harayonowei, të pënɨ si
taponomihe. Si ponomihe, kuonomi yai. 

Raharariwë a ha përɨikunɨ, hapa yami a kuoma, Raharariwë, ɨ̃naha pë
kuonomi, pë horohoonomi, yami të pata përɨoma, Raharariwë, ɨ̃hɨnɨ hawë
kamanɨ haxɨrɨ ha a kuama mai! Huxomi ha hei yahi mɨ kurenaha, mau u
makure, ɨ̃naha e mɨ huxomi kuoma. Hãto e nahi kuoma, huxomi hamɨ. 

Hei kurenaha, hii te hi kɨ ĩtatawë rë kurenaha ɨ̃hɨ naxi si kɨ ha yahipɨ
kuoma. Naxi si kɨ tõtahima ha nahi ĩtapoma, õkapoma. Ɨhɨnɨ si kɨ tapoma.
Si kɨ kãi përɨoma, Raharariwënɨ. Ɨhɨ a yai. Si rë paramarenowei, si rë
wëyënowehei, ɨ̃hɨnɨ si poma, hapa. Ai të iha ai si keo mao tëhë, ɨ̃hɨ si
kɨ ha pëkɨ tana yãoma. Ɨhɨnɨ si kɨ hãto nahi ĩtapoma. Kamiyë pëma kɨ no
mori preaama, ɨ̃hɨ a mao ha kë kunoha, Raharariwënɨ si kɨ tapou mao ha
kunoha, napë ya wãha harɨnɨ: 

Naxi ya hĩ kɨ puhii! Ya kuimi. Naxi ya si kea harayou. Pëma kɨ mori
kunomi. Ɨha si kɨ rë piyërenowehei, Raharariwënɨ si ha piyërënɨ, e si ha
piyërëhenɨ, naxi pëma a waɨ. Si hore wamou praukurayoma. Ɨnaha. Ɨha, pë
tëë kɨpɨ rë yupɨrenoweinɨ si kɨ tapoma. Naxi si kɨ tapoma. Ɨha rë e si
kërema. 

Ɨhɨ weti a ha nomohori ha nakarënɨ huxomi hamɨ weti e warokema? Kama
Raharariwë e nahi hamɨ kama pë tëënɨ a kãi rukërayoma, huxomi hamɨ. Hei
a rë rukëre pë tëënɨ a kãi nomohori rë rukërayoma. Hei kurenaha pë tëënɨ
kihi naha a kãi upraoma, kihi tokori kɨ hena rë kurahari naha. Ɨhɨ kihi
rë të u pata. Të u parɨkɨ pata. Huxomi hamɨ, mɨ wakarawëmi makui, tëë e
upraoma. Ɨhɨnɨ a rë yurenowei, a nomohori kãi kirimapë pë hɨɨ iha. A kãi
waroa he yatikema, a rë kirimaɨwei. Ɨhɨ rë a wãha Omawë. Uxuweimɨ si pë
wai rë riyëhëi ɨ̃hɨ a wãha kuoma, ai kiritamɨ mai! Ɨhɨ a rë kuinɨ pë tëë
e yai rë yurenowei ɨ̃hɨ e yai. Omawënɨ e yurema. Raharariwë tëëpɨ. Ɨhɨ a
kãi ha rukërɨnɨ, a kãi kea he yatiparioma, hũka kɨ ha kahurënɨ,
hesiopɨnɨ hũka kɨ tapoma. Hũka kɨ ha kahurënɨ: 

--- Kopou! Ũi, ũi, ũi, ũi, ũi, ũi! --- a kuɨ he yatiyo hërɨma --- Wa mixi
tuo mai kë të! --- e kuma. 

Hɨ̃ɨɨ! Tëɨ! 

Hei kurenaha, hei kurenaha e kutarioma. Kihi hawë hũria ana pë pata wakë
kõre rë përɨkɨi, kihi hawë werehi hakosi pë pata ĩyë rë xurixuripraroi,
yahipɨ mɨ kuoma. Katehe nahi ha a përɨoma, a përɨopë ha e warokema. Pë
tëë e ã hama. 

--- Hape! Siohahë kë a! --- e kuma --- Siohahë ya kãi huɨ kuhe, ya yurema! 

E mamo pata xatitarioma. 

-Ëhë rë të? --- e pata kutarioma --- Hɨ̃ɨ! Ëhë rë të? 

--- Hape, hẽaroyë kë a. Ya yurema, a ma rë riëhëi --- a noã tama. 

E të kahe pata watëkarioma. Heinaha kama e kuma. 

Kihi kë të ko pë pata hẽraxa urerewë hawë nasi kɨ pë pata ureremoma. E të wãrima, pë tëë iha, ɨ̃hɨ ei rë a kãi rë keparuhe iha. 

--- Naxi rë si pata ë, exi të si kɨ pata? 

Mihi te hi rë kurenaha, ɨ̃naha kuwë, no aiwë te hi ha, katehe te hi ha,
hawë yahetipa kë kɨ sipo rë kure te hi pata praopë ha pë xɨɨ a kuopë ha
e kãi tikëkema. 

--- Mihi kana hi ha xoe hẽarohë a kãi ta tikëtaru! Mihi hi ha! --- e pata
kuma. 

Iwaiwa kana hi pata ha, iwaiwa kana hi ha a tikëmakema. A kirimapë. Ɨhɨ
iwa rë pë rë kuprouwei naha, e te hi pata kuproma, hi taɨ ma mahei rë. A tikëkema. 

--- Hei hii rë hi kɨ? 

--- Ma, hei hii hi kɨ mai -- e kuma hesiopɨ. Të pë pata nohi makepraɨ. 

--- Hii kɨ mai! Naxi si kɨ! --- hesiopɨ e kuma. 

--- Naxi si kɨ! --- e kuma. 

Pëma si pë wãha yuapë. Ɨhɨnɨ si pë wãha hirakema. 

-Ɨhɨ weti naha të pë taɨ? 

--- Kihi ko ha kë anɨ, a ha rãanɨ, a waɨ --- e kuma. 

--- Hoaa! --- e kuma. 

Heinaha si kɨ kararu pata rë yëtëawei, e të kɨ kararu pata rë yërëkëi, e
hoyorema. 

--- Kero! Ipa ya si kɨ keapë! --- e ha kunɨ, a wã wayomama, nomohori. 

A mori wama. Rahararinɨ a mori suhama, hẽaropɨ mori suhama, tëë e mao ha
kunoha, pë tëënɨ a he rumapoma, a tikëmapou rëoma, pei waku ha. 

--- Kahënɨ ware wa yurei, ya tikëkei pei wa waku ha -- e kuma, hẽaropɨ.

Ɨnaha të pë tapehe, a tikëmarema. Ɨnaha të pë tama, të pë taɨ hiraɨ ha. 

A kirimarema, e te hi pata aketeprarioma, amokɨ kerarioma, a ëaëmoma,
Omawë. E kiriri ɨ̃kɨma. 

Ɨhɨ e te hi pata hokëhokëmou xoapario ha, kõpina e yëtëkema, tutu si wai
rë auwei, ɨ̃hɨ e të si xi harirawë tutusiprarioma, e kõpinaprarioma.
Kutaenɨ a no preapraroma. 

Pë tëënɨ a nokarema. 

--- Hape! Ɨnaha ta tihë! Ipa a rii, ya yurema yaro! Ɨnaha a taɨ mai! Ɨnaha
wa kuaaɨ tëhë, ya kãi kuomi. Pei, pëhë kɨ ta kõo hërɨ! --- e kuɨ
xoaoma. 

Pei e nahi parɨkɨ mɨa pëmarema. A mori waɨ waikita ha, a kãi tokurayo
hërɨma. A kãi tokurayoma, a mori suhaɨ ha. 

--- Tu, tu, tu, tu, tu, tu, tu! --- e kuɨ waikia ha. 

Ɨhɨ Rahariwë iha si rë yurenowei, si kãi ha pokëprarunɨ, kama a rë
kuawei ha, si hipëke herɨma. Si hipëke herɨma. Ɨnaha si rë takenowei,
ɨ̃hɨ rë pëma tẽhi pë waɨ, pëma kɨ ohii maopë, pëma kɨ pehi yurema. 

Ɨhɨ si rë yurenowei ai ihirupɨ ai, ai si tëëpɨ, maxi hamɨ e si ɨ̃kɨɨ
hëoma. Si ɨ̃kɨma. Pei si xi ɨ̃kɨɨ hëoma. Pei si kãi yai ma tokurɨhe ha. 

--- Hapemi! Hapemi! --- heparapɨ e kuɨ hëkema. A mɨa no pou hëkema, ai a
yupɨre herɨɨ ha, ɨnaha e kuma. Ɨnaha të tamahe. 

\chapter{O dilúvio}

\section{O ressurgimento dos Yanomami e o aparecimento dos napë}

Depois da morte do irmão e do sobrinho, Yoahiwë e Omawë fugiram rio
abaixo. Havia somente um rio, o rio Tanape. Eles encontraram, no
percurso, outro sobrinho, filho de Manakayariyoma, cuja mãe se
considerava a irmã dos dois, por ter o mesmo nome que a irmã deles. 

Como se deu esse encontro? Enquanto os dois estavam no meio do rio, eles
escutaram um chamado vindo de cima. 

\textit{To}! \textit{To}! Escutaram um som descendo na direção deles. \textit{Hɨ̃ tuuuuu}! Fazia o
bebê descendo na direção deles. O sobrinho desceu na sua direção com
sede. \textit{Hɨ̃ɨɨ tëɨ}! A criança estava amarrada em uma haste de palmeira.

--- \textit{Ũa, ũa, ũa}! Tio! Tio!

Acabava de descer ao chão, sentado no meio daquela haste de palmeira:

--- Sede! Sede! Tio! Sede!

Eles o agarraram. Não foi gente que deu à luz esse bebê. Ele não
tinha pai gerador, mas apenas apareceu nessa haste de palmeira. 

Levaram-no, muito sedento. Ninguém o gerou! Levaram-no, pois era o
sobrinho. 

Foi lá que os dois encontraram o filho de Manakarariyoma. Por causa
desse bebê encontraram o grande rio fechado. Os humanos foram
exterminados por causa daquele bebê sedento que surgiu do nada. 

Como? Omawë não abriu sem razão essa água na qual se afogaram os
Yanomami; foi por causa do sobrinho cujo fôlego se apagava. Ele morria de
sede. 

Como se deu esse evento?

--- \textit{Ẽ, ẽ, ẽ}! --- assim ele respirava. 

Vendo a criança arfar, com a moleira se esvaziando, o tio chorava por
causa da sede do bebê. 

--- Não vou deixar meu pequeno sobrinho morrer! Não vou ficar feliz,
não! --- dizia chorando e recuando. 

Os olhos do bebê estavam virando e Omawë dava a espuma de sua baba para
ele beber. O bebê chupava, mas chupava em vão. Sua boca se enganava com a
urina do tio. Ao final, o irmão mais novo refletiu e decidiu conseguir água de qualquer forma. 

Ele previu que a água estava guardada embaixo de uma pedra. \textit{Tuku, tuku, tuku}! A água batendo debaixo da pedra fazia esse som. 

A pedra era muito dura, estava bem fincada; mesmo assim ele conseguiu
tirar um pouco, não de vez. Ele suspendeu a pedra só um pouco,
inclinando-a para o lado, não por baixo. Assim que ele a empurrou um pouco, a
água jorrou. O bebê, que já estava morto, ressuscitou por causa dessa
água.

\textit{Tuuuuuuuu}! Fez a água. 

A água logo jorrou longe. O jorro caiu lá, bem em baixo. A água se
curvou e escondeu o céu. A água jorrou durante duas noites e o rio
encheu rapidamente. Por causa da sede daquele bebê, a gente daquela
época desapareceu. 

Foi depois desse evento que nossos antepassados surgiram; a água levou
os mortos lá pra baixo, onde os dois preparavam redes. Omawë, Yoasiwë e o
sobrinho fizeram uma cerca de paxiúba dura como ferro fincado na água,
para reter os mortos levados pelo rio. No mesmo lugar onde Omawë e
Yoahiwë se localizavam, eles fizeram outro jirau bem forte para moquear
os mortos e fazer aparecer os \textit{napë}. 

Feito isso, a água já trazia outras pessoas sofrendo, os afogados. O rio ficava bem estreito no meio da terra dos \textit{napë}. Pegaram os
mortos lá bem perto da terra dos \textit{napë}, lá em baixo. Chegavam
pelas águas os quase mortos, onde estavam os dois. 

O que fizeram Omawë e seu irmão? Eles usaram aquelas redes que haviam
tecido. O irmão mais velho, que era mais esclarecido, disse: 

--- Irmão menor! Faça logo, os meus já estão passando! Meu irmão, os
primeiros já estão passando! 

Prepararam corda de embira. Os dois teceram um tipo de tarrafa transparente.
Eles salvaram as pessoas com tarrafa de embira \textit{omaoma}. Não paravam de
lançar a rede onde chegavam os mortos, pegando um a um. Puxaram a rede
como se fossem peixes, muitos peixes. 

Feito isso, ele e o irmão mais novo os jogaram em cima do jirau, um por
um, quando o fogo grande se erguia; eles os assaram como se fossem
caça. 

\textit{Xãaaai}! O fogo crescia com a gordura derretida de gente. O irmão
mais novo cortou as folhas novas de sororoca e as deixava no chão. Em
cima dessas folhas, juntava os corpos cremados, enquanto o outro
recuperava os corpos no mesmo instante. 

O sobrinho, que rapidamente cresceu, ajudava seu tio. 

Eles os jogavam em cima das folhas, os corpos cremados. O irmão mais
velho recuperava os corpos e os raspava. 

Ele raspava os corpos cremados com um tipo de colher grande. \textit{Xoe}! \textit{Xoe}!
\textit{Xoe}! A pele dos cremados produzia esse som. 

Eles jogavam os corpos cremados em cima das folhas novas de sororoca,
que estavam no chão uma perto da outra. 

Aqueles que estavam bem raspados, eles separavam, pegando-os com um
tipo de arpão. \textit{Tuuuuuuuu}! Eles não erravam. 

--- \textit{Aë, aë, aë, aë, aë, tãrai}! --- faziam os dois assim. --- \textit{Aë, aë, aë,
pei kë oooo}! \textit{Tãrai}! 

Fizeram assim levantar os Yanomami ressuscitados com flechas na mão. 

--- Todos, todos, todos, todos! \textit{Tãrai}! De pé! --- disse Omawë.

Primeiro ressurgiram os Yanomami e depois apareceram os \textit{napë}. 

Somente os Yanomami Horonamɨ se ergueram com as flechas na mão. 

--- \textit{Kia}! \textit{Kia}! \textit{Kia}! \textit{Ha, ha, haaaa}! --- diziam eles. --- \textit{Tãrai, ha asi
ɨ̃ɨɨ}! 

Eram gordos, pintados e enfeitados, com as penas de cauda de arara,
altivos.

--- \textit{Ha, ha, ha, ha}! 

O irmão mais velho riu sem parar dos que estavam se transformando. Olha
aqui! As mulheres púberes se erguiam elegantes, apesar de terem morrido
afogadas, elas reapareceram como moças novas. 

Depois de ter amontoado a metade dos corpos cremados, eles os jogaram na
água. Os mortos não caíram na água em silêncio. 

O primeiro a dizer: \textit{Ĩxima}! \textit{Ĩxima}! se tornou piranha. \textit{Ĩxima}! \textit{Ĩxima}! Todos
que disseram isso se tornaram piranhas. 

--- \textit{Koooorooouuu}! \textit{Kuxu}, \textit{kuxu}, \textit{kuxu}! --- faziam. 

Os demais se transformaram em matrinxã. Apareceu um monte de peixes
flutuando. A superfície da água ficou completamente coberta e sumia de
tanto peixe, a água não se mexia mais de tanto peixe, de cuja carne gostamos
tanto.

Quando dizemos:

--- É um pacu! --- na verdade, comemos a carne de gente que se tornou
peixe, comemos a carne preta de Yanomami. 

Da nossa carne de Yanomami surgiram os \textit{napë}. Os \textit{napë} surgiram a partir dos Yanomami que se transformaram. A partir dos nossos corpos cremados, com a transformação, os \textit{napë} surgiram. 

Antes mesmo da existência dos \textit{napë}, já havia aparecido a moradia
deles, antes do surgimento dos motores, do canto do galo, e antes dos
próprios ancestrais. Embora sejam a origem de tudo isso, antes da
existência dos \textit{napë} e das \textit{napëyoma} falando uma língua
estrangeira, Omawë e Yoasiwë se surpreenderam quando ressuscitaram e
rasparam os outros. 

Os ressuscitados ficavam de pé, elegantes, e atiravam flechas para
provar que estavam sãos e salvos.

--- Irmão menor, aqueles que eu queria fazer surgir, eu os
deixei à deriva na água! Veja o resultado! --- disse o irmão maior,
porque havia ressuscitado todos eles. 

Omawë se transformou e, onde se transformou, a imagem dele se
tornou \textit{napë}. Ele criou os \textit{napë}. Lá a sua imagem se misturou
e ainda está lá, eles não a veem. 

--- Eu sou aquele que ressuscitou vocês! --- ele diz isso? Omawë não diz
isso. Ele existe ainda, ainda está vivo em algum lugar. --- Sou aquele que
ressuscitou vocês! --- Omawë não nos diz isso. 

Não fica perto onde eles estão morando. Os dois que realizaram essa
ressurreição são os que deram origem aos \textit{napë.} Apesar de os dois
ficarem lá longe em seguida, outros \textit{napë} chegaram. 

O que aconteceu? Se vocês forem pra lá, vocês não chegarão. Os dois
foram até os \textit{napë} de pele vermelha. 

O rio abaixo fica dentro da terra. Como qualquer rio, a cabeceira da
água nunca fica baixa, quando as fontes das águas se juntam, elas
descem; assim as águas do dilúvio se juntaram e formaram o mar. O rio
não desce plano, o rio acaba e entra na terra. 

Os dois que fizeram a transformação moram lá, além dessa parte do mundo,
rio abaixo, lá onde emboca a mãe do rio, onde entra um rio só. Eles são
eternos. 

Onde você pode ir? Não há aonde ir. Lá fica o rio, onde não há floresta,
onde não há nada. Daquele lado vivem os dois, que fizeram a grande
transformação. 

Os \textit{napë} se espalharam. Lá, eles se reproduziram, rio abaixo; não
há mais ninguém onde ressuscitaram. Eles se dividiram rio abaixo. Foram
morar em outros rios. Sim. Foi assim, nenhum dos que viviam
antes sobreviveu. 

Quando o rio levou os Yanomami, sobreviveram somente dois xaponos. Os
antepassados renasceram e se desenvolveram. Eles sobreviveram para
sempre. 

As montanhas altíssimas se chamavam Ũaũaiwë, Rapai e Wãima. Os
sobreviventes conseguiram subir até o cume da serra Rapai, se agruparam
como carapanãs, como moscas, e estavam tristes. 

A serra Ũaũaiwë acabou afundando; restou somente o cume da montanha. O céu parecia se apoiar no cume das montanhas Rapai e Wãima. Não deu para
o rio atingir o cume dessas montanhas. 

\chapter{Yanomamɨ pë rë kuprarionowei}

\section{Napë pë rë kuprarionowei}


Ɨhɨ pata katehe ya wãha rë taprare a no pëxɨrɨ ha pata kɨ rë tokupɨre,
huxuoripɨ, kɨpɨ ha karërɨnɨ, ɨ̃hɨ u mau tahiawë u pata kuo parɨoma, mahu
të u pata kuo parɨoma, Tanape u pata mahu kuo parɨoma, Tanape u ha a
pehi rërërayo hërɨma. Ɨhɨ të ha kɨ rë tokupɨre, weti naha të tapɨpë mai!
Weti naha kɨ kupɨapë? Kɨ rë rërëpɨɨ he tiherire, kama kɨpɨnɨ puhinɨ
masiri a ĩtaa taprapë ha, u ĩsitoripɨ ka he rë karopɨpraɨwei, kɨ
rërërayou kuhe, ɨnaha kɨ kupɨrayo hërɨma. 

Ɨhɨ kɨ rë hupɨrɨhe, kɨpɨ huxuoripɨ kurayo hërɨma, oxe a no pë xɨrɨ ha,
Horonamɨ a no pë xɨrɨ ha yai, katehe a kuoma yaro, kihamɨ kɨ rĩya ha
tokupɨnɨ, napë no patama rë pëtamare hamɨ, ɨ̃hamɨ kɨpɨ rĩya ha
napëpɨpronɨ, kama parimi kɨpɨ rĩya ha kupɨpronɨ, kɨ kupɨrayoma. Ɨhɨ kɨpɨ
rë hupɨre hamɨ, hei hekamapɨ e pëpɨtario hërɨma. Ɨhɨ rë a no kãi tapɨo
kiriopë. Ɨhɨ rë hekamapɨnɨ Yanomamɨ të pë mixi rë tukenowei, të pë napë
rurukema. Ei a wã amixiri ha. Hekamapɨ oxeoxe e pëpɨtarioma. Ɨhɨ exi të
ihirupɨ kuoma? Hekamapɨ e yai rë pëpɨtarionowei, kɨpɨ mɨ amopɨprou tëhë,
e të pehi kãi hõra pëpɨtarioma. 

\textit{To}! \textit{To}! Ɨhɨ rë pehi kãi nihõraɨ waikio kuimati. \textit{Hɨ̃ tuuuuu}! E pehi kãi
nihorapë kurati. Ɨhɨ kɨpɨ yaɨpɨ rë kuonowei, Manakayariyoma yaɨ e wãha
kupɨoma. Manakayariyoma, ɨ̃hɨ ihirupɨ yaɨ e wãha ha hiropɨnɨ, hekamapɨ e
pehi kãi itopɨa nokarayoma, amixiri. \textit{Hɨ̃ɨɨɨ tëɨ}! Si he wai õkawai si ha,
ɨ̃hɨ kama si ha. 

--- \textit{Ũa, ũa, ũa}! Xoape! Xoape! --- e kuma. 

E pehi kãi ma kei tutoranɨ rë. Mɨ amo si yai ha e të wai tirereranɨ:

--- Amixi, amixi! --- e kuma. --- Xoape! Amixi! --- e kuma. 

A yupɨre hërɨma, a hurihipɨa nokare hërɨma. 

Yanomamɨ të pë haikiamapë, ɨ̃hɨ ei e amixiri xomi pëpɨtarioma. Yanomamɨ
tënɨ a ha keprarɨnɨ, e hekamapɨ pëpɨtonomi. 

Ɨhɨ Omawë yaɨpɨ wãha Manakayariyoma kuoma. E ha rarakɨhenɨ, pë hɨɨ e
kuami makui, e pehi kãi paye kirioma. E amixiri wai hayure hërɨma. E ha
rarakɨhenɨ mai! Ɨhɨ hekamapɨ e yaro, a yupɨre hërɨma. Ɨhɨ iha ɨ̃hɨ
Manakayariyoma ihirupɨ iha motu u he hapɨre kirioma, u ka pata rë
kahuaɨwei ha, ɨ̃naha a tama. Hapa Yanomamɨ pë mixi rë tuamanowei, kama u
xomi ukëpraɨ pëonomi. 

Kama hekamapɨ a rë kui a nomarayou ha, mixiã wai makei ha, amixi pënɨ. 

--- \textit{Ẽ , ẽ, ẽ}! --- të mixiã wai ku hërɨɨ ha. 

Të amipɨ wai prokea hërɨɨ ha, pë xɨɨnɨ a mɨa no yupoma, amixiri. 

--- Ipa ya të wai amixiri hesi wai keamaɨ mai kë të! Pë ã wa no hore nohi
wëaɨ totihiopë! A mɨa no pou ha --- e kuma, pata e mɨa kãi tiruroma. 

E mamo kɨ si wai no mɨhou ha, pei e kahi u pë moxi yãxaamou yaro, pei
kahi u moxi makui ha, hekamapɨ e të wai xomi karereoma, kahikɨ no
preaamama. Pei naasi pë ha, kahikɨ no preaamama. Ɨhɨnɨ u he rë ukëprara
kiri hamɨ, hekamapɨ e nomarayoma makui, ɨ̃hɨ unɨ e harorayoma. E të mixiã
wai maa totihikema makui, e harorayoma. Yakumɨ oxe ke e puhi yaitou he
ha yatironɨ, pë oxenɨ e u napë pata yëa he yatikema. A huxuo yaro. 

--- Ipa ya të wai amixi no preaamaɨ ta tawë, asi yai ɨ̃ɨɨ! --- e ha kunɨ, e
u he ukëpraɨ puhiopë yaro, e u napë pata yëa xoakema. 

\textit{Tuku, tuku, tuku}! E u pata rë tihekitaaɨwei, e u pata kuma. 

Masiri a koro hesika kɨ ha, u pata rë tihekitihekimaɨwei, masiri kë a
mori pata yutumaɨ tëhë, maa ma pë pata hiakawë rë kurenaha kuwë, masiri
e pata xatio kirioma, ɨ̃naha kuwë, a ukërema, wãisipɨ, pruka ukëpraɨ
haikionomi, ĩsitoripɨ e të pata yëpɨhɨtarema. Ɨnaha të pata kaɨhɨhɨ taɨ
kurayoma, ëyëmɨ, ei korokoro të mɨ hamɨ mai! Ɨnaha të pata hirurutaɨ
kurayoma. 

U he ukëpraɨ tëhë: 

--- \textit{Tuuuuuuuuu}! --- e u pata kurayoma. 

Të u mohororopɨ pata xatia xoaketayoma. Kihamɨ katiti. Kihi u pata rë
hirekera kiri, kihi të parɨkɨ hamɨ, kihamɨ u ora pata kerayoma, u ora
pata keye kirioma, u mohe pata hoteye kirioma, kihi të parɨkɨ pata maa
xoarayo hërɨma. Marɨ hërɨnɨ, ɨnaha të kɨpɨ titi wai kupɨa kure, yëtu të
u pata orayoma, të u pata ha orɨnɨ, kamiyë pëma kɨ nohi patama napë pë
no patama rë kui, pë rë kuprore, pë yãkapɨ kiriopë, pë pakakumama. 

Hei kurenaha të pë rë kui të pë no rë preaanowei, motu unɨ të pë rë
yurenowei, manaka pë rë kure naha, sipara kohikɨ, ɨ̃hɨ sipara kɨ yai rë
xororoi, hei kurenaha arana kɨ taprarema. Pë rë makepɨpraaɨwei, hiakawë
tehi pë uprahawë ɨ̃hamɨ napë pë urihipɨ pë rë hõrimanowei hamɨ, ɨ̃hɨ kɨ rë
kupɨare ha, yëtu kihi naha kama pë pehi taprarema. 

Ha taprarɨnɨ, yëtu kihamɨ ai Yanomamɨ të pë rë yuatayouwei, të pë no
preaama. Napë a urihi rë kure të pëɨxokɨ hamɨ, u pata wãkokoa, hei
kurenaha ahetewë inaha u pata kua. Ɨhɨ të rë kure ha, ɨ̃ha pë hõrimama,
ai u hamɨ mai, ɨ̃hɨ u mahu ma rë kure. Napë a urihi ora a he te heha, pë
yãkapɨ kirioma. Hëyëha pëɨxokɨ ha mai! Ɨha kɨpɨ rë rëpɨre ha, Yanomamɨ
të pë waharo nomawë waropraama, mau u hamɨ. 

Ɨhɨ exi të taprarema? Kama kɨpɨ napëpɨprou puhio yaro, no patapɨ pë
napëpropë, ɨ̃hɨ exi të tiëpɨprarema? Exi të kɨ tiëwanɨ të pë yãkapɨma?
Kama pata e rë kui e puhi ha moyawëonɨ: 

--- Oxei! Taprakɨ! Ipa ai pë hayuo waikipe ë! --- e kuma --- Pusi ipa ai pë
ora hayuo rë waikipe ë! --- pata e kuma. 

Omaoma asita nakɨ kopepɨprarema. Yɨɨ wararawë kurenaha të kɨ
tiëpɨprarema. Omaoma asita nakɨ ka ha, pë yakapɨma. Hei yuri wama pë
hora rë yakaɨ rë xoaotiowei. Ɨhɨ wama të pë hõra rë taɨwei naha, të kɨ
pata ha tiëpɨprarɨnɨ, të kɨ pata xëyëapɨɨ mɨ paoma, të kɨ pata pahërapɨɨ
mɨ paoma. Pruka yuri kurenaha, të pë pehi kãi pata xaixaimapɨma. 

Kuaaɨ tëhë, oxenɨ hei kanare kɨ pata ha pë xëyëpɨpraama, pë pahërapɨma,
pata kaɨ wakë upraupramou tëhë, pë iximapɨma, yaro kurenaha. 

\textit{Xããããi}! Yanomamɨ të pë pata rë tapɨi, të pë tapɨ pata tamama, të pë wakë
pata upraupramomapɨma. Kihinaha ɨ̃hɨ oxenɨ ketipa, tukutuku hena kɨ pata
pëararɨnɨ. Hei ketipa hena kɨ pata rë kui, henakɨ pata prapɨparema, ɨ̃hɨ
henakɨ ha pë ĩxi hirapɨpraama; ainɨ pë yãkaɨ tëhë, pë yãkama. 

Patanɨ pë nokamaɨ tëhë, ɨ̃hɨ hekamapɨ e kãi rë kui pë xɨɨ a payerimama.
Ɨha e kãi pataa haɨtarayoma. 

Ɨha të pë pata makepɨpraɨ, të pë pata ĩxi pahërapɨma, pahërapɨma, ɨ̃hɨ
patanɨ pë ma yãkaranɨ makui, pë hõama, ĩxi. Pata trëmɨ kurenaha kuwë të
patanɨ pë ĩxi hõama. 

\textit{Xoe}! \textit{Xoe}! \textit{Xoe}! Të pë si pata tapɨmama. Taɨ tëhë, ɨ̃hɨ ketipa e henakɨ
pata rë praawei ha, tukutuku xĩro, henakɨ he pata rë tikëawei ha. 

Hei pë rë hõare, hei a rë amohõroprarɨhe kiha a xëyëaɨ. A ha hurënɨ, ɨ̃hɨ
a pehi kãi ha hurihirënɨ. \textit{Tuuuuuuuu}! A hãsɨkano nomawë xëyëaɨ kateheo
maopë. 

--- \textit{Aë, aë, aë, aë, aë, tãrai}! --- e pë kutoma. -- \textit{Aë, aë, aë, pei kë oooo}!
\textit{Tãrai}!

Pë xerekapɨ temɨtemɨ pë kãi uprahatamama. 

--- Haikia, haikia, haikia! Tãrai, upra! 

Wãiha napë pë pëtopë. Waiha ɨ̃hɨ Yanomamɨ të pë makui të pë ĩxi, napë pë
pëtopë waiha, napë a rë kui kama pë rii no patapɨ pëtopë. Ɨhɨ hapa
Yanomamɨ të pë pata xĩro xɨrɨkou parɨoma. 

Hei Yanomamɨ Horonamɨ hei të pë rë kui hei xereka rë a pata kãi xɨrɨkou
parɨo kupohei. Ɨnaha të pë kuaama. 

--- \textit{Kia, Kia, Kia, ha, ha, haaaaaa}! --- të pë pata kutoma --- \textit{Tãrai, ha asi
ɨ̃ɨɨ!} --- Të pë tapɨ, hei kurenaha, pë mɨ kãi yãpramoma, të pë kãi
pauximoma. Të pë arapɨ xina kãi pata xɨrɨkamama. 

--- \textit{Ha, ha, ha, ha}! 

Pata e kuaprarou pëoma, pë rë hõrimaɨwei. Ma rë kui! Suwë të pë rë kui,
mokomoko suhe mo kɨ, të pë pata xɨrɨkoma no aiwë. Heinaha kuwë a mixi ma
tukenowei makui, mokomoko a pëtarioma, yɨpɨmono tute. Ɨhɨ pë ĩxi rë kui,
pë ha pëɨxokɨpɨtarɨnɨ, pë ĩxi keamapɨma. Mamikai mai! 

Hapa a rë kui, a rë ĩximaĩximamohe, yëkëri pë hupë. Ɨhɨ yëkëri rë pë
kuprario kuhe. \textit{Ĩxima kë! Ĩxima kë!} A rë kuhe, kamiyë pëma kɨ no ɨ̃hɨpë
iximi tikowë, hei pëma kɨ kuyëhëwë, wa rë ĩxiwei, ɨ̃hɨ yëkëri wa
kuprario. 

--- \textit{Kooorooouuu}! \textit{Kuxu, kuxu, kuxu}! --- pë rë kutonowei. 

Hei tepaharimɨ pë rë kui, a wãno rë auwëpraruhe, a ĩxi rë keamare,
tepaharimɨ a hurayou. Hawë yuri të kɨ pata kuo yaioma, kama kɨ kuoma.
Hawë hẽkahẽka të kɨ pata kupro hërɨpë, pei të u pata maamaɨ totihioma.
Ɨhɨ të pë yãhi kɨ rë kui pëma të pë totihiaɨ. 

--- Auyakari yai! --- wa kuu makui, Yanomamɨ pë rë yuriprarionowei, pëma pë
waɨ, ɨ̃xi Yanomamɨ pëma të pë yãhi kɨ ĩxi waɨ. Yanomamɨ të pë yãhi kɨ
ĩxi! 

Kamiyë pëma kɨ yãhi kɨ rë kui, pëma kɨ yãhi kɨ napëprarioma. Kamiyë pëma
kɨ yãhi kɨ rë kui, ɨ̃hɨ hei kama pë yãhi kɨ napë. Yanomamɨ të pë rë
hõrimanowei, kama napë pë kuprarioma. 

Kamiyë pëma kɨ yãhi kɨ ĩxi, hõrimano napë pë kuprarioma. Ɨhɨ të mao pë
ha, napë a përɨo mao pë ha kɨ kupɨa makure, ɨ̃ha napë pë yahipɨ pata kua
xoaparioma. Ɨha rë të pë yahipɨ pata xɨrɨkou nokamoma, pehi kãi
rërëarewë pë kãi kuonomi. 

--- Kakara ëëë! --- të pë rë kuuwei, të ã kãi hirionomi makui, ɨ̃hɨ hei rë
të pë no porepɨ rë pëtore ha, të pë rë hãsɨkapɨre ha, të ã pata rë
pëprou mɨ tarɨa xoare ha, a ãtiprarioma, kɨ kiripɨrarioma, kamanɨ të pë
taɨ makure, napë ɨ̃naha pë kuonomi makui, suwë napëyoma të pë rẽaa ka
pata kuparioma. 

Të pë no pata aiwë xɨrɨkoma, të pë xerekapɨ ã pata no mɨhɨta parioma. 

--- Ma rë kui, ɨ̃naha oxei, ɨ̃naha ipa ya pë rĩya ha tapranɨ, ya pë
yarëkëyoruu kuhe. Ei ma rë kui, pë ta mɨ! --- pata a kuma, të pë
hõriprarei yaro. 

Ɨhɨ të pë hõrimapë hamɨ hekura Hõrimawëteri pë hiraa kure. Kama a
hõriprarioma, ɨ̃hɨ të pë hõrimapë ha, kama a no uhutipɨ rë
napëprarionowei, kamanɨ pë taprarei yaro, ɨ̃hamɨ a no uhutipɨ rë
nikereprarionowei, a kua xoaa, a tapraimihe. 

Ɨhɨ kamiyënɨ pëma kɨ rë hõrimanowei, hei kë ya! A kuɨ? A kuimi. Kama a
kua xoaa, a temɨ xoaa, koyokoa xoaa. Kamiyënɨ hei pëma kɨ yai rë hõrimaɨ
totihiohe, hei kë ya! Pëma kɨ noã taimi. 

Kɨ yai rë kupɨre, hei kama ɨ̃hɨ të aheheha mai! Kama kɨpɨnɨ të pë rë
hõrimanowei, ɨ̃hɨ napë kamanɨ pë rë taprarenowei, yami kɨpɨ kupɨye
kirioma makui, e pë rë warokenowei. 

Ɨhɨ weti naha të kua kure? Ɨhamɨ wama kɨ huokema, wama kɨ hawëɨ xoaa.
Napë Wakëweiteri pë iha, kɨpɨ warokema. Ɨhɨnɨ napë kama e pë warokema. 

Pei u koro pata titia. Hei u rë kui, kihi u koro hu hërɨpë hamɨ u koro
he yarɨhɨwë kua taomi, heinaha ɨ̃hɨ u koro he kõkapropë ha, të u pata
nihõroa. 

Ɨhɨ të pata maxi hamɨ Yanomamɨ hõrimarewë kɨ kupɨye kirioma. Hei u rë
kui, kama pë nɨɨ u koro keopë ha tahiawë u koro pata titipropë ha. 

Weti hamɨ wa hu hërɨpë? Huimi. Hëyëmɨ u pata koro xĩro kuwë, ai a urihi
kua rë mare hamɨ, yawëtëwë hamɨ. Ɨhɨ të maxi hamɨ Yanomamɨ hõrimarewë kɨ
kupɨa. 

Napë pë huo xoaokema. Ɨhɨ tëhë, ɨ̃ha të pë pata paraa xoarayoma, koro
hamɨ pë huokema, pei pë hõrimapë ha, ai pë kuami. Koro hamɨ, kihamɨ, pë
tihetirarioma, ai u pë hamɨ pë huokema, pë kuprawë. Awei, ɨnaha të
kuprarioma: hapa të pë rë kuonowei ai të hëpronomi. 

Motu unɨ të pë rë yurenowei, tahiawë porakapɨ të pë yahipɨ wai
hëprarioma. Hei të pë rë hëprarionowei, hei të pë no patama rë kui, të
pë no patama raropë, pëtopë. Pë hëaaɨ hëopë. 

Hei pei kɨ he yai rë toreonowei, Ũaũaiwë kɨ wãha pata kuoma, hei Rapai
kɨ, hei Wãima kɨ. Rapai kɨ hemono hamɨ kurenaha të pë ha tuikutunɨ,
ɨ̃naha të pë hëprou kurayoma, ai të hëprou ma mare ha, hei kɨ xĩro rë
tirepii hamɨ, kihamɨ kɨ ora pata hɨ̃kɨhɨotayoma. Porakapɨ të pë wai
hëprarioma. Hei naha të kɨ hesikakɨ wai rë kuprouwei hamɨ, hei kurenaha
të wai ha, ukuxi kurenaha, moomoo kurenaha të pë wai ha kõkaprarutunɨ,
të pë wai hëtarioma. Të pë puhi okii makure. Ɨha u pata hawërayotayoma.

Ɨhɨ Rapai kɨ pata hamɨ, hei Wãima kɨ pata hamɨ, Wãima pei ma kɨ wãha.
Hehu tirewë totihiwë kɨ hëpɨprarioma. Kihamɨ, Rapai kɨ yai kua kure.
Kihamɨ Wãima kɨ yai kuwë he torewë. 

\chapter{O surgimento da primeira mulher}
 
Agora vem a história dos Unissexuais. O nome deles, os Unissexuais, vem
do fato de a mulher não ter aparecido imediatamente. Eles se agruparam. 

O surgimento dos nossos antepassados aconteceu a partir da perna de
Japu. Ainda não havia surgido a mulher. 

Depois do rapto da filha de Raharariwë, segue a história dos Unissexuais
que moravam juntos, quando não havia mulheres. 

Apesar de serem homens, eles faziam sexo entre eles. Apesar de terem
pênis, eles faziam sexo entre eles. No meio da copulação
deles, surgiu Japu. 

Depois do surgimento de Japu, lá os ancestrais dos Waika, dos Xamatari,
dos Parahiteri e dos Xirixianateri nasceram. Na parte carnuda abaixo do
joelho da perna de Japu, apareceu uma vagina. 

Nossos antepassados não nasceram da perna de Japu. O surgimento dos
nossos ancestrais é outra história. 

Nessa época, os antepassados dos Waika se reproduziram a partir da
perna de Japu. Nossos antepassados e, consequentemente, nossas gerações
já nasceram de mulher e não a partir da perna de Japu. Foram outros
os antepassados dos Waika. Os antepassados dos Waika nasceram da vagina da
filha de Japu. A história deles é diferente.\footnote{O par \textit{waika}/\,\textit{xamatari} parece ter sido usado originalmente para designar outros grupos yanomami vivendo em região geográfica diversa de quem fala, os primeiros ao norte e oeste, e os segundos ao sul, reconhecendo-se neles conjuntos de características que os particularizam. Os termos foram atribuídos em diferentes momentos pelos brancos para designar grupos específicos de forma estável e, no caso de \textit{xamatari}, para designar a própria língua do tronco yanomami usada pelos Parahiteri que fizeram este livro.} 

Japu apareceu e se misturou a eles. Aquele que ia ser o marido
dele já morava no xapono. Depois de a vagina aparecer, foi ele quem
falou com o marido para fazer sexo. 

A vagina apareceu, semelhante àquela das mulheres. Os Unissexuais se
satisfizeram com a perna de Japu. Daí, nasceu a primeira mulher, o que
possibilitou aos Unissexuais fazerem sexo. Foi graças à perna de Japu, portanto,
que eles se satisfizeram. 

Nossos primeiros ancestrais são oriundos da perna de Japu. Primeiro,
nasceu uma mulher, depois nasceu outra. Assim se fez. Depois, outra.
Assim. Fizeram outra. Depois de nascer a primeira mulher, a partir da
qual nasceram as outras, surgiram os parentes de nossos antepassados. 

--- \textit{Prohu}! --- logo disse o homem que nasceu primeiro. 

Chamaremos o primeiro homem que nasceu assim de nosso antepassado.
Eles se multiplicaram a partir da mulher que nasceu da perna de Japu, a
partir da filha mais velha de Japu. Continuaram a se multiplicar.
Nasceram cinco mulheres. Nasceram assim. 

Depois de elas nascerem, a vagina sumiu da perna de Japu, porque ela já
havia feito as mulheres. Já estavam se reproduzindo. Nós os chamamos de
nossos parentes. Nossos antepassados se reproduziram, continuaram a se
reproduzir. 

Se não fosse a perna de Japu, nossos parentes não existiriam. Aqueles
com os quais nos misturamos e fazemos amizade são nossos parentes,
nossos verdadeiros parentes. Foi o que aconteceu.

Depois de nascerem, eles ocuparam toda a floresta. Não são outros que
nos fizeram! Não foi Omawë que nos criou! Omawë mora em cima, apesar de
ter morado primeiramente nesta floresta. Ele fugiu da condição de
Yanomami. Ele voou, ele foi morar lá em cima, assim eram os dois irmãos
no início. Foi assim mesmo. 

Nossos ancestrais saíram da perna de Japu; ele não morou mais ali, foi a
um lugar diferente. Japu se chamará Napërari, quando se tornar
espírito. Seu marido também.

Apesar de ter um pênis igual a nosso, ele fazia sexo com outro homem,
apesar de ter o pênis amarrado, ele engravidou a perna de Japu, gerando
as mulheres, na perna dele mesmo. A vagina na perna menstruava, e ficava
sentada no chão no tempo da menstruação, ensinando a sentar no chão em
período menstrual. Fez aparecer a barriga na perna. 

Depois de nascer, a mulher chorou, o pai se levantou rapidamente, ele a
pegou logo, cortou o cordão umbilical e ela cresceu rapidamente. Naquele
momento, Japu se tornou homem novamente. Foi assim. Essa história
acabou.

\chapter{Suwë a kuprou rë hapamonowei}


Ɨhɨ të rë kui hamɨ, ai të ã kuprou piyërayoma. Posinawayorewëteri pë rii
rë kure. Kama Posinawayorewë teri pë wãha rii kuoma, suwë a kuo haɨonomi
yaro. Kama Posinawayowëteri pë rii rë hiraonowei. 

Xĩapo mata hamɨ. Suwë a kuonomi. 

Ɨhɨ Raharariwë tëëpɨ ha yëpɨrënɨ, ɨ̃hɨ të nosi weti hamɨ, Posinawayorewë
teri pë hiraoma, suwë a kuami yaro. 

Hei kurenaha, wãro makui të pë posi na wayoma. Të pë moroxi kuprawë
makui, të pë posi na wayoma. Kuopë ha, Xĩapo wama a wãha rë hiripouwei,
a pëtarioma. 

A ha pëtarunɨ, kihamɨ hei Waika, Xamatari ai pë, Parahiri pë, Xirixiana
teri pë, pë no patama rii rë keowei, Xĩapo mata hamɨ, të nakahikɨ
kukema, pei mata xĩapɨ hamɨ. 

Xĩapo mata hamɨ, kamiyë pëma kɨ no patama wawëɨ taonomi. 

Ai Waika pë rë kui ɨ̃hamɨ pë no patama rii rarorayoma, ɨ̃hɨ Xĩapo mata
hamɨ. Kamiyë pëma kɨ rë kui, Xĩapo mata hamɨ kamiyë pëma kɨ no patama
wawëɨ taonomi. Ai, Waika pë rë kui, ɨ̃hamɨ pë no patama rii harayoma, ɨ̃hɨ
Xĩapo mata hamɨ. Ɨhɨ Xĩapo mata no tëëpɨ naka hamɨ Waika pë no patama
rii rë wawërayonowei, ɨ̃hɨ kama e të ã rii yaiwehe. Kama të ã. 

Ɨhɨ a ha pëtarunɨ, Xĩaporitawë wãro a nikereoma. Hẽaropɨnɨ të pë kãi
përɨoma. Ɨhɨ iha e ã hama, e nakahikɨ ha kuikunɨ, suwë kurenaha, nakahi
kɨ kukema. 

A ha kopeprarɨnɨ, ɨ̃hɨnɨ Posinawayorewëteri pë mɨ takema, Xĩaponɨ, Xĩapo
mata hamɨ, pë kuwëmomama, Xĩaponɨ pë mɨ takema. 

Ɨhɨ mata hamɨ pëma kɨ no patama kupropë, a keprarema, suwë hapa. Ɨhɨ a
rë kuprore hamɨ, suwë, ɨ̃hɨ të nosi yau hamɨ ai suwë. Ɨnaha a tama. Ɨhɨ
të hamɨ ai a suwë. Ɨnaha. Ai a rë taɨ kõrahei, ai suwë. Ɨnaha pë kepraɨ
kurayoma, suwë. Ɨhɨ ei pë rë kui hamɨ, hapa a rë patare hamɨ, kamiyë
pëma kɨ no patama maxi kupropë, a wawërayoma. 

--- \textit{Prohu}! --- a kuɨ haɨtaoma, wãro, hapa a rii kukema. 

Të rë kuprore hamɨ, kamiyë pëma kɨ nohi patama, pëma kɨ kupë. Hapa wãro
a keprarioma. Xĩapo mata hamɨ a suwë rë keprore, hapa a rë taare hamɨ,
naka hamɨ, pë raro hërɨma. Ɨhamɨ pë rarou xoao hërɨma. 

Ɨnaha pë keprou ha kuronɨ, suwë, pei pë rë keprare hamɨ, ai, ai, ai, ai,
ai, pëma kuɨ. Ɨnaha pë kepraɨ kurayoma. Pë rë kuprore hamɨ, kama mata
nakahikɨ rë kui maprarioma. Suwë pë taa waikikema yaro. Pë paraɨ
waikitao hërɨma. Ɨhɨ kamiyë pëma kɨ maxi, pëma kɨ kuɨ. Pëma kɨ no patama
pararayoma, para hërɨma, paraa xoarayoma. 

Ɨhɨ Xĩapo mata hamɨ a mao ha kunoha, pëma kɨ maxi përɨhɨwëmi. Ɨhɨ ai
pëma kɨ nohimayou rë nikerei, ɨ̃hɨ pëma kɨ maxi. Ɨhɨ pëma kɨ yai maxi.
Ɨnaha të kuprarioma. 

Kuprarunɨ, urihi kutarenaha, pë përɨaɨ kuprario hërɨma. Ai tënɨ pëma kɨ
ha taprarënɨ mai! Omawënɨ pëma kɨ ha pëtamarɨnɨ mai! Omawë kama a rë kui
heaka hamɨ a kuwë yaro, hëyëha a përɨoma makui, hei a urihi ha a përɨoma
makui Yanomamɨ a tokurayoma. A yërayoma. A heakaprarioma. Ɨnaha kɨ
kupɨoma. Ɨnaha të yai kuwë. 

Xĩapo mata hamɨ pëma kɨ rë hare, nohi patama, kamiyë pëma kɨ. Ɨhɨ pëma
kɨ rë keprarenowei, Xĩapo mata hamɨ pëma kɨ no patama rë rarorayonowei
hamɨ, ɨ̃ha a kãi përɨo taonomi. Yai hamɨ a rii kurayoma. Kama Napërari a
wãha hekura kuopë. Hẽaropɨ xo. 

Kamiyë pëma kɨ mo kuo rë kure naha a kuwëmou, mo hãhoa makure, pë
taprarema, ɨ̃hɨ mata yai hamɨ, mata ha xipënarɨnɨ, mata na kãi ĩyëama,
maito rooma, të pë hiraɨ ha, makasi kɨ kãi wawërayoma. 

A ha keprarɨnɨ, suwë a ũaũamoma, pë hɨɨ e itoprarioma, a hurihia
nokarema, xi kãi hanɨprarema, rope e kãi patarayoma. Ɨhɨ tëhë a wãroprou
kõrayoma. Ɨnaha të kuprarioma. Ɨhɨ të ã rë kui, të ã maprarioma. 