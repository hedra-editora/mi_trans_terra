\textbf{Os comedores de terra} \textls[15]{apresenta histórias como o surgimento da maniva, a mandioca. E o dilúvio no qual Omawë --- irmão de Horonamɨ, grande pajé que surgiu dele mesmo e junto com as florestas ---, está envolvido e que deu origem aos brancos, os \textit{napë}. \textit{Os comedores de terra} faz parte do segmento Yanomami da coleção Mundo Indígena --- com \textit{O surgimento da noite}, \textit{A árvore dos cantos} e \textit{Os comedores de terra} ---, que reúne quatro cadernos de histórias dos povos yanomami contadas pelo grupo Parahiteri. Trata-se da origem do mundo de acordo com os saberes desse povo, explicando como, aos poucos, ele veio a ser como é hoje.}

\textbf{Anne Ballester} foi coordenadora da \textsc{ong} Rios Profundos e conviveu vinte anos com os Yanomami do rio Marauiá. Trabalhou como professora na área amazônica e atuou como mediadora e intérprete em diversos \textit{xapono} do rio Marauiá --- onde também coordenou um programa educativo. Dedicou-se à difusão da escola diferenciada nos \textit{xapono} da região, assim como à formação de professores yanomami, em parceria com a \textsc{ccpy} Roraima, incorporada atualmente ao Instituto Socioambiental (\textsc{isa}). Ajudou a organizar cartilhas monolíngues e bilíngues para as escolas yanomami a fim de que os professores pudessem trabalhar em sua língua materna. Trabalhou na formação política e criação da Associação Kurikama Yanomami do Marauiá e participou da elaboração do Plano de Gestão Territorial e Ambiental (\textsc{pgta}), organizado pela Hutukara Associação Yanomami e o \textsc{isa}.

\textls[18]{\textbf{Mundo Indígena} reúne materiais produzidos com pensadores de diferentes povos indígenas e pessoas que pesquisam, trabalham ou lutam pela garantia de seus direitos. Os livros foram feitos para serem utilizados pelas comunidades envolvidas na sua produção, por isso uma parte significativa das obras é bilíngue. Esperamos divulgar a imensa diversidade linguística dos povos indígenas no Brasil, que compreende mais de 150 línguas pertencentes a mais de trinta famílias linguísticas.}



