\textbf{Os comedores de terra} reúne narrativas que abordam o surgimento de outros elementos do mundo natural e social dos Yanomami. São apresentados, aqui, as histórias dos comedores de terra, a origem e os feitos dos demiurgos Horonam\i e seu irmão, Omamë, o surgimento da maniva, a mandioca, e o dilúvio no qual Omamë está envolvido, que deu origem aos brancos, os \textit{napë}.

\textbf{Anne Ballester} nasceu em 1955 na França e viveu por 24 anos com os Yanomami. Enquanto ativista, trabalhou como agente de saúde no combate à malária, além de alfabetizadora em língua yanomami e professora de português para jovens e adultos em posições de liderança indígena. É cofundadora da \textsc{ong} Rios Profundos. 

\textbf{Coleção Mundo Indígena} reúne materiais produzidos com pensadores de diferentes povos indígenas e pessoas que pesquisam, trabalham ou lutam pela garantia de seus direitos. Os livros foram feitos para serem utilizados pelas comunidades envolvidas na sua produção, e por isso uma parte significativa das obras é bilíngue. Esperamos divulgar a imensa diversidade linguística dos povos indígenas no Brasil, que compreende mais de 150 línguas pertencentes a mais de trinta famílias linguísticas.



