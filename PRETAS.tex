\textbf{Os comedores de terra} \textbf{A árvore dos cantos} reúne histórias contadas por pajés yanomami do rio Demini, sobre os tempos antigos, quando seres que hoje são animais e espíritos eram gente como os Yanomami de hoje. Estas histórias contam como o mundo veio a ser como ele é agora. Trata-se de um saber sobre a origem do mundo e dos conhecimentos dos Yanomami que as pessoas aprendem e amadurecem ao longo da vida; por isto, é um livro para adultos. 

\textbf{Anne Ballester} nasceu em 1955 na França viveu por vinte e quatro anos com os Yanomami. 
Enquanto ativista, trabalhou como agente de saúde no combate à malária, foi alfabetizadora em língua 
yanomami e professora de português para jovens e adultos em posições de liderança indígena. É cofundadora da \textsc{ong} Rios Profundos. Atuou como tradutora e organizadora dos livros \textit{A árvore dos cantos}, \textit{O surgimento dos pássaros}, \textit{O surgimento da noite} e \textit{Os comedores de terra}, todos incluídos na Coleção Mundo Indígenas.

\textbf{Coleção Mundo Indígena} reúne materiais produzidos com pensadores de diferentes povos indígenas e pessoas que pesquisam, trabalham ou lutam pela garantia de seus direitos. Os livros foram feitos para serem utilizados pelas comunidades envolvidas na sua produção, e por isso uma parte significativa das obras é bilíngue. Esperamos divulgar a imensa diversidade linguística dos povos indígenas no Brasil, que compreende mais de 150 línguas pertencentes a mais de trinta famílias linguísticas.



