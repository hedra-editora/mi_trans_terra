\chapterspecial{O {surgimento} {da} {maniva}}{}{}
 

{Q}{uem pegou} e espalhou a maniva para nós comermos? Não havia maniva.
Nossos antepassados não possuíam maniva como a que nós plantamos. Não
tinham maniva, ela não existia. 

No início, quando o monstro Raharariwë morava com suas filhas, sem outros
parentes, seu xapono ficava no fundo da água. Raharariwë não morava em
terra seca. No início, ele tinha sua casa dentro da água. 

As manivas estavam em uma casa que tinha a estrutura fincada, como a
nossa. Ela tinha esteios de maniva fincada. Ele possuía maniva. Morava
com elas, Raharariwë, ele mesmo. No início, era ele que a possuía, a
maniva que foi multiplicada e dividida. Quando ninguém tinha plantação
de maniva, ele amarrava sua rede lá onde ela estava. Ele guardava essa casa
de maniva fincada. Se não fosse ele, nós sofreríamos, se Raharariwë não
possuísse a maniva. 

--- Vou plantar maniva. --- A gente nem diria isso. A partir de
Raharariwë, a maniva se espalhou e nós comemos farinha. A maniva se
espalhou como alimento. Ele possuía a maniva, aquele cujas filhas foram
pegas. Omawë a quebrou e a pegou.

Quem foi traiçoeiramente chamado para dentro dessa casa? Quem chegou? A
filha entrou com ele dentro da casa de Raharariwë. Ela
entrou maliciosamente com ele dentro da água. A filha estava lá de pé.
Olha só a água! Olhe a superfície da água! A filha estava de pé dentro
da água escura. Ela o levou maliciosamente, para que o pai o assustasse.
Ele assim conseguiu chegar até Raharariwë, mas só para passar medo. 

Chamava"-se Omawë. Ele tinha o nome daquela saíra verde tão bonita, não
de outro passarinho! Foi ele mesmo, Omawë, que levou a filha de
Raharariwë. Ele conseguiu entrar na água depois de fechar as narinas,
foi sua mulher que as fechou:

--- Kopou! Ũi, ũi, ũi, ũi, ũi, ũi! --- ela fez assim. --- Você não se
afogará! --- disse ela. 

--- Hɨ̃ɨɨ! Tëɨ! 

O xapono logo ficou sem água. A casa estava pintada de vermelho escuro como
as frutas \emph{hũria} bem maduras, como o vermelho cor de sangue das
asas de papagaio em movimento. Raharariwë morava em uma casa bonita e
Omawë chegou até lá. A filha lhe falou:

--- Pai! É teu genro! Eu trouxe teu genro, eu casei! 

Raharariwë olhou para Omawë. 

--- Aquele é teu? Hɨ̃ɨɨ! É teu?

--- Pai! É meu marido! Eu casei! Ele é bonito, tu não achas? 

Ele sorriu. Lá, aquelas raízes saídas da terra pareciam enormes.

Onde o sogro estava, havia um pau grande, enfeitado e bonito, no chão,
parecido com a pele das costas de um poraquê. A filha sentou com Omawë.

--- Vai te sentar naquele pau com teu marido, filha
querida! 

Ele o fez sentar em cima do pau \emph{iwaiwakana} para assustá"-lo. Esse
pau se mexeu como se mexem os jacarés. Ele o fez sentar em cima de um
pau que se mexia por si só. Omawë se sentou e perguntou a ela: 

--- Isto é madeira?

--- Não é madeira! É maniva! São manivas! --- disse a mulher. 

Nós chamaremos de maniva, foi ela que ensinou o nome maniva. 

--- Como se faz aquilo? 

--- Ele quebra para fazer beiju, ele faz beiju e come --- respondeu
ela. 

--- Hoaa! --- exclamou. 

Ele tirou algumas mudas de maniva que estavam juntas:

--- Kero! Eu plantarei minhas manivas! --- Omawë disse reservadamente à esposa. 

Raharariwë quase comia o genro, se a filha não o protegesse,
guardando"-o sentado no seu colo. 

--- Tu me pegas e me fazes sentar no teu colo. 

E assim outras farão, fazer sentar o marido no colo. Assim ela fez e
ensinou a fazer. 

Omawë ficou com medo por causa do pau que se mexia, ficou assustado,
gritou e chorou como uma criança. Chorou de medo. 

Como o pau estava se mexendo, Omawë se tornou caba e grudou ao pau, se
transformou em caba \emph{tutu si}, daquelas branquinhas. Ele ficou
abalado. 

Ela conteve o pai: 

--- Pai! Não faça isso! É meu, eu casei! Não faça isso, por favor! Se
você fizer isso, eu não ficarei aqui! Vamos voltar, nós dois!

Ela mostrou a frente da casa. Raharariwë quase comeu Omawë, por isso ela
fugiu com ele. Ela também fugiu, pois Raharariwë quase comeu o marido. 

--- Tu, tu, tu, tu, tu, tu! --- ele já estava dizendo.\footnote{   Aqui há um jogo com o verbo \emph{tu-}, cozinhar em água.}
%Jorge, por favor verifique se funcionou a nota que inseri aqui 

Aquele que levou as manivas de Raharariwë boiou com elas e as deu aos
seus parentes. 

Nós comemos o que ele deixou para acompanhar nossa comida, para não
sentir mais fome, ele pegou para nós comermos. 

Ele levou as mudas de maniva, cujos brotos ficaram chorando. Já havia
realmente fugido. 

Levaram manivas cujos irmãos ficaram chorando. 

--- Pai! Pai! --- choravam assim. 

Assim fizeram.